% ******************************************************** %
% Copyright © Jonathan Gitlin 2011 – 2013
%
% This work is licensed under the Creative Commons 
% Attribution-NonCommercial-ShareAlike 3.0 Unported License
%
% For More Information See the Creative Commons Website
% http://creativecommons.org
%
% For A human readable summary, See:
% http://creativecommons.org/licenses/by-nc-sa/3.0/
%
% For the Full Text of the License:
% http://creativecommons.org/licenses/by-nc-sa/3.0/legalcode
%
% No Claim Is Made To Original Works 
% Of Other Sources Cited Within This Work
%
% ******************************************************** %

\chapter{Preface}

The genesis of this book is difficult to pin down, exactly.  It was not one precise moment or idea.  Rather, it evolved over a period of time.  If I had to look back, though, and pick a moment that I could point to and say ``there, that's the beginning of this book,'' it would probably be sometime in late 2006 or early 2007.  I was in my second year in law school, and enrolled in a class on Constitutional law.  We had covered many typical Constitutional law topics in the class, which I'm sure scores of students had covered before in their Constitutional law classes and which I'm sure many more will cover for years to come.  But there came a point at which I remember looking up and thinking ``why am I just \textit{now} learning this?''  

I had had a United States elementary, middle, and high school education.  I had a bachelor's degree from an accredited liberal arts college.  I had a master's degree from an accredited and nationally recognized university.  I had a number of years in the work force under my belt, and had participated in many national, state, and local governmental elections.  And yet, I had (at best) only the vaguest notion of the vast majority of the fundamental concepts that we were covering in the class.  And that seemed strange to me.

While I loved just about every part of my Constitutional law class, probably my favorite part was when our professor would say ``everybody put your pens \& pencils down, stop typing on your laptops, and just listen.''  It was story time.  Our professor, who seemed able to recite entire chapters of history books on command, would tell us the story of the United States at the point in time relevant to whatever case we were discussing.  And then, after he had given us the history, we would discuss the case.  

What made this so educational was that in telling us the story, our professor gave each case a frame of reference.  He gave it context.  And the context gave it meaning.

And this lead to a curiosity on my part for a greater understanding of the context of the Constitution and the United States government as a whole.  And that lead to the Federalist papers, which lead to biographies of the founding fathers, which lead to anthologies of United States history, which lead to etc, etc, etc.

Four years and numerous cases, articles, and books later, I found myself jotting down skeleton outlines of the various parts of the United States government for my own reference.  And gradually those outlines grew and grew until... well, here we are.

The purpose of this book is to provide an accessible reference book for the basic functions of the machine that is the United States government.  It is by no means meant to be a complete or exhaustive treatment of the subjects contained herein.  It is, rather, meant to be a primer, hopefully leading to more exploration and learning.  If any or all of the topics in this book interest you, you can use the cited references as jumping off points to learn more.

In the end, I hope this serves as an interesting reference and a spark to readers' interest.

In the spirit of those that strive to provide a greater understanding of the amazing creation that is the United States, I dedicate these pages to all those who want to know more.
