% ******************************************************** %
% Copyright © Jonathan Gitlin 2011 – 2013
%
% This work is licensed under the Creative Commons 
% Attribution-NonCommercial-ShareAlike 3.0 Unported License
%
% For More Information See the Creative Commons Website
% http://creativecommons.org
%
% For A human readable summary, See:
% http://creativecommons.org/licenses/by-nc-sa/3.0/
%
% For the Full Text of the License:
% http://creativecommons.org/licenses/by-nc-sa/3.0/legalcode
%
% No Claim Is Made To Original Works 
% Of Other Sources Cited Within This Work
%
% ******************************************************** %

\chapter{Executive Branch}

\section{Purpose of Chapter}
This chapter covers the Executive Branch of the United States government, that branch of the government that executes and enforces the laws.

\section{Creation}
Article II creates the Executive Branch.  Per Article II, Section 1:

\begin{quote} 
The executive power shall be vested in a President of the United States of America.
\end{quote}

\subsection{Members of the Executive Branch}
In creating the Executive Branch, the Constitution vests all executive power in a single person -- the President.\footnote{Article II, Section 1 (``The executive power shall be vested in a President of the United States of America.'').}  However, all of the laws and government of the United States cannot be executed by a single person.  Thus, the President is expected to have subordinates underneath him as part of the Executive Branch.  The Constitution recognizes this by providing for inferior members of the Executive Branch.\footnote{Article II, Section 2 (``he may require the opinion, in writing, of the principal officer in each of the executive departments''); Article II, Section 2 (``He shall have power, by and with the advice and consent of the Senate … all other officers of the United States, whose appointments are not herein otherwise provided for, and which shall be established by law''); Article II, Section 2 (``Congress may by law vest the appointment of such inferior officers, as they think proper, in the President alone, in the courts of law, or in the heads of departments'').}

In practice, the Executive Department is divided into various departments covering different subject matters, such as the Department of Defense (which covers the military and defense of the United States), the Department of State (which covers foreign relations with other nations), and the Department of Treasury (which handles the finances and money of the United States).  The heads of these departments are known as ``secretaries,'' such as the Secretary of Defense, Secretary of State, and Secretary of Treasury.

\section{Election}

The Constitution establishes the method of electing the President in Article II, Section 1.

\subsection{Electoral College}
The President is elected by the Electoral College, which is made up of Electors who cast their vote for the President.  The States appoint the Electors, and each State is entitled to appoint a number of electors equal to the number of Representatives plus the number of Senators the State has in Congress.\footnote{Article II, Section 1 (``Each state shall appoint, in such manner as the Legislature thereof may direct, a number of electors, equal to the whole number of Senators and Representatives to which the State may be entitled in the Congress: but no Senator or Representative, or person holding an office of trust or profit under the United States, shall be appointed an elector.'').}
So, for example, if Texas has 36 Representatives in the House of Representatives, then Texas would be entitled to appoint 38 electors -- which is equal to 36 (the number of Representatives) plus 2 (the number of Senators).

Additionally, even though the District of Columbia is not a State, it also appoints electors for the Presidential election.  Per the 23rd Amendment, the District of Columbia is entitled to appoint Electors in the same manner as States -- a total number of Electors based on (1) the number of Representatives it \textit{would} be entitled to if it were a State plus (2) the number of Senators it would be entitled to if it were a State.\footnote{Amendment 23 (``The District constituting the seat of government of the United States shall appoint in such manner as the Congress may direct: A number of electors of President and Vice President equal to the whole number of Senators and Representatives in Congress to which the District would be entitled if it were a state, but in no event more than the least populous state; they shall be in addition to those appointed by the states, but they shall be considered, for the purposes of the election of President and Vice President, to be electors appointed by a state; and they shall meet in the District and perform such duties as provided by the twelfth article of amendment.'').}

\subsection{Voting Process}

\subsubsection{Timing}

Per Federal law, the states have to appoint electors for the Presidential election once ever four years, on the first Tuesday that follows the first Monday in November.\footnote{3 USC \S\ 1 (``The electors of President and Vice President shall be appointed, in each State, on the Tuesday next after the first Monday in November, in every fourth year succeeding every election of a President and Vice President.'').}

\subsubsection{How Electors Are Appointed}
Each State sets the method of appointing its electors by its state's law.  For example, in Texas the Electors are voted on by the citizens of Texas, and the Texas Election Code lays out the method and manner of selecting the Electors by popular election.\footnote{See Title 11 of the Texas Election Code, especially Chapter 192.}

\subsubsection{Voting}
Once the Electors are chosen by the States, they meet in their respective states and vote by ballot for the President and the Vice President.
Per the 12th Amendment, the Electors vote separately for two different offices -- the President and the Vice President.  Each of the Electors gets one vote for each office.\footnote{Amendment 12 (``The electors shall meet in their respective states and vote by ballot for President and Vice-President, one of whom, at least, shall not be an inhabitant of the same state with themselves; they shall name in their ballots the person voted for as President, and in distinct ballots the person voted for as Vice-President, and they shall make distinct lists of all persons voted for as President, and of all persons voted for as Vice-President, and of the number of votes for each, which lists they shall sign and certify, and transmit sealed to the seat of the government of the United States, directed to the President of the Senate;'').}

\subsubsection{Party Tickets}
This mechanism where the electors vote separately for the President and the Vice President is a result of the 1800 election.  
Under the Constitution's original presidential election mechanism, the Electors cast 2 votes each, without a distinct vote for the President or Vice President.  Under the original scheme, the candidate with the highest number of electoral votes became the President and the next highest was the Vice President.\footnote{Article II, Section 1 (``The electors shall meet in their respective states, and vote by ballot for two persons, of whom one at least shall not be an inhabitant of the same state with themselves. And they shall make a list of all the persons voted for, and of the number of votes for each … and the votes shall then be counted. The person having the greatest number of votes shall be the President, if such number be a majority of the whole number of electors appointed … and if there be more than one who have such majority, and have an equal number of votes, then the House of Representatives shall immediately choose by ballot one of them for President … In every case, after the choice of the President, the person having the greatest number of votes of the electors shall be the Vice President.'').}

Unlike today, at the Constitution's inception the concept of a ``party ticket'' with explicitly designated Presidential and Vice Presidential candidates was unknown.  In fact, the very concept of a ``political party'' (or ``faction'' as it was called back then) was anathema.  Political parties were seen as sordid associations and considered to be destructive to national coherence and unity.\footnote{See, for example, George Washington's farewell address, where he warned the country about the dangers that factions posed (``all combinations and associations, under whatever plausible character, with the real design to direct, control, counteract, or awe the regular deliberation and action of the constituted authorities, are destructive of this fundamental principle, and of fatal tendency. They serve to organize faction, to give it an artificial and extraordinary force; to put, in the place of the delegated will of the nation the will of a party, often a small but artful and enterprising minority of the community; and, according to the alternate triumphs of different parties, to make the public administration the mirror of the ill-concerted and incongruous projects of faction, rather than the organ of consistent and wholesome plans digested by common counsels and modified by mutual interests.'').}

For the first two presidential elections, the Constitution's original mechanism presented no serious problems.  This was mostly because George Washington was running for the office, though, and no else was seen as a serious contender.  However, when George Washington declined to run for a third term in the 1796 election, cracks began to show in the system.   Because the Constitution's original presidential election system didn't provide a method of voting for a ``party ticket,'' candidates from two different parties were elected President and Vice President in 1796.  John Adams, a Federalist, was elected as President, and Thomas Jefferson, a Republican, was elected Vice President.  And even though the concept of ``faction'' still carried a tinge of baseness, by this point political parties had emerged as solid institutions in the United States, even if they weren't always openly acknowledged.

It wasn't until the 1800 election, however, that serious cracks began to show in the system.  By this point, political parties were running, informally, pairs of candidates for the Presidency and Vice Presidency, in what we would today consider to be a ``party ticket.''  In the 1800 election, the Republican party ran Thomas Jefferson and Aaron Burr as, ostensibly, its presidential and vice-presidential candidates against the Federalist party's ``ticket'' of John Adams and Charles C. Pinckney.  After the votes were cast, the Republicans had won the election.  However, when the electoral ballots were actually counted, Thomas Jefferson and Aaron Burr had tied for the number of electoral votes, 73 to 73.

While everyone implicitly understood that Thomas Jefferson was the Presidential candidate and Aaron Burr the Vice Presidential candidate, Jefferson and Burr were tied in votes for the office of President and per the Constitution the election went to the Federalist-controlled House of Representatives for a run-off.

Jefferson expected Burr to gracefully bow out, but Burr's ambitions took hold, and he did not withdraw.
The House went through thirty six run-off votes until finally the deadlock was broken and Jefferson was selected as President.
Thereafter, the 12th Amendment, wherein Electors vote separately for the President and the Vice President, passed Congress on December 9, 1803 and was ratified by the States on June 15, 1804.

\subsection{Winning}
The Electors cast their votes by ballots, and make a list of everyone that they voted for for both President and Vice President, including the number of votes for each person.  The Electors then seal their list and send it to the President of the Senate.\footnote{Amendment 12 (``they shall name in their ballots the person voted for as President, and in distinct ballots the person voted for as Vice-President, and they shall make distinct lists of all persons voted for as President, and of all persons voted for as Vice-President, and of the number of votes for each, which lists they shall sign and certify, and transmit sealed to the seat of the government of the United States, directed to the President of the Senate;'').}
When the Electors' votes arrive, the President of the Senate opens the lists and counts the votes in the presence of both houses of Congress.\footnote{Amendment 12 (``The President of the Senate shall, in the presence of the Senate and House of Representatives, open all the certificates and the votes shall then be counted;'').}
Once the votes are counted, the winner is the candidate with a majority of the electoral votes.\footnote{Amendment 12 (``the person having the greatest number of votes for President, shall be the President, if such number be a majority of the whole number of electors appointed;'').}
For example, supposing that there are a total of 538 electoral votes, to win the election a candidate would need 270 votes 
\begin{math}
(538 \div 2 = 269, 269 + 1 = 270)
\end{math}

\subsection{Run Off}
If no candidate receives a majority of the electoral votes, then the election goes into a run off between the top three candidates.\footnote{Amendment 12 (``and if no person have such majority, then from the persons having the highest numbers not exceeding three on the list of those voted for as President, the House of Representatives shall choose immediately, by ballot, the President'').}  However, the runoff election is decided by the House of Representatives, not the Electoral College.\footnote{Amendment 12 (``and if no person have such majority, then from the persons having the highest numbers not exceeding three on the list of those voted for as President, the House of Representatives shall choose immediately, by ballot, the President'').}
The votes for the runoff are taken State by State in the House of Representatives, and each State only gets a single vote, regardless of the number of representatives it has in the House.\footnote{Amendment 12 (``But in choosing the President, the votes shall be taken by states, the representation from each state having one vote;'').}
Once the votes are counted, the candidate with a majority of the votes wins.\footnote{Amendment 12 (``a majority of all the states shall be necessary to a choice.'').}

A runoff for the Vice President is decided in a similar manner, but by the Senate instead of the House of Representatives.  Unlike a Presidential run off, though, in the Vice Presidential run off each Senator's vote counts individually, and only the top two Vice Presidential candidates are considered instead of the top three.\footnote{Amendment 12 (``The person having the greatest number of votes as Vice-President, shall be the Vice-President, if such number be a majority of the whole number of electors appointed, and if no person have a majority, then from the two highest numbers on the list, the Senate shall choose the Vice-President; a quorum for the purpose shall consist of two-thirds of the whole number of Senators, and a majority of the whole number shall be necessary to a choice. But no person constitutionally ineligible to the office of President shall be eligible to that of Vice-President of the United States.'').}

\subsection{Trivia About Why We Have Electoral College}
While seemingly complex and intricate when compared to a simple popular vote, the electoral college has been the United States' method of choosing its President and Vice President since the inception of the country.
In the constitutional convention, various methods of electing the President were considered, including popular election.
A discussion of why the convention ultimately chose the Electoral College mechanism is beyond the scope of this work.  
However, as a pure bit of trivia, in James Madison's notes from the constitutional convention dated July 19, 1787, while discussing the possibility of a parliamentary prime-minister-style appointment of the president, Madison describes why, in his opinion, the electoral college was ultimately the best choice: 
\begin{quote}
He was disposed for these reasons to refer the appointment to some other source. The people at large was in his opinion the fittest in itself.  It would be as likely as any that could be devised to produce an Executive Magistrate of distinguished Character. The people generally could only know \& vote for some Citizen whose merits had rendered him an object of general attention \& esteem. There was one difficulty however of a serious nature attending an immediate choice by the people. The right of suffrage was much more diffusive in the Northern than the Southern States; and the latter could have no influence in the election on the score of the Negroes. The substitution of electors obviated this difficulty and seemed on the whole to be liable to fewest objections.
\end{quote}

\section{Qualification}
The Constitution sets out several restrictions on who can be president.  To be eligible to be President, a person must:

\begin{enumerate}
\item Be a natural born citizen, as opposed to a naturalized citizen.\footnote{Article II, Section 1 (``No person except a natural born citizen, or a citizen of the United States, at the time of the adoption of this Constitution, shall be eligible to the office of President'').}
\item Be at least 35 years old.\footnote{Article II, Section 1 (``No person ... shall ... be eligible to that office who shall not have attained to the age of thirty five years'').}
\item Have been a resident in the United States for at least 14 years.\footnote{Article II, Section 1 (``No person ... shall ... be eligible to that office who shall not have ... been fourteen Years a resident within the United States'').}
\end{enumerate}

\section{Term of Office}

\subsubsection{Length of Term}
The President is elected for a four year term that begins at noon on January 20th of the year following his election and ends on noon of January 20th four years later.\footnote{Article II, Section 1 (``He shall hold his office during the term of four years''); Amendment 20 (``The terms of the President and Vice President shall end at noon on the 20th day of January, and the terms of Senators and Representatives at noon on the 3d day of January, of the years in which such terms would have ended if this article had not been ratified; and the terms of their successors shall then begin.'').}

\subsubsection{Term Limit}
As originally enacted, the Constitution had no limit on the number of terms a person could serve as President.  However, George Washington as the first President established an informal practice of serving only two terms.  After George Washington this tradition endured from the 18th Century all the way to the first half of the 20th Century, as no President ran for or served more than 2 terms during that period.  However, in 1932 in the throes of the Great Depression, Franklin Roosevelt was elected President, and went on to run for and get elected three more times, in 1936, 1940 and 1944, as World War II broke out and America was drawn into the fighting.  However, despite Roosevelt's unprecedented success at the polls, in 1947, three years after Roosevelt was elected for a fourth term, Congress passed and the States ratified the 22nd Amendment, which set a limit on the number of times a person could be elected to the Presidency.

Per the 22nd Amendment, a President is restricted to 2 elected terms, or 1 elected term if he has previously served as President for more than 2 years of someone else's term.\footnote{Amendment 22 (``No person shall be elected to the office of the President more than twice, and no person who has held the office of President, or acted as President, for more than two years of a term to which some other person was elected President shall be elected to the office of the President more than once.'').}  This sets a maximum limit of 10 years that a person could constitutionally serve as President (two years during someone else's term plus two four-year terms of that person's own).

\section{Oath of Office}
Every President must take the following oath before assuming the office:

\begin{quote}
I do solemnly swear (or affirm) that I will faithfully execute the office of President of the United States, and will to the best of my ability, preserve, protect and defend the Constitution of the United States.\footnote{Article II, Section 1 (``Before he enter on the execution of his office, he shall take the following oath or affirmation:--``I do solemnly swear (or affirm) that I will faithfully execute the office of President of the United States, and will to the best of my ability, preserve, protect and defend the Constitution of the United States.'''').}
\end{quote}

\section{State of the Union}
Article II, Section 3 requires the President to give a report to Congress on the condition of the country.  This report is known as the ``State of the Union'' address.

\begin{quote}
He shall from time to time give to the Congress information of the state of the union, and recommend to their consideration such measures as he shall judge necessary and expedient
\end{quote}

The Constitution requires that the President to give the address ``from time to time.'' In practice, the President delivers the State of the Union speech once a year, usually in late January or early February.

During an event such as the State of the Union when the United States' leaders are all gathered in a single place, the Executive Branch will designate an individual in the line of presidential succession to remain away from the meeting in a secure location so that should a catastrophic event occur that killed or incapacitated all of the other members of the Executive Branch a new President in the line of succession could legally assume the office and the Executive Branch of the United States would continue to function uninterrupted.


\section{Powers}
The President has two types of powers: (1) explicit powers and (2) implicit powers.  Explicit powers are those powers that are explicitly conferred on the President, either via the Constitution or by law.  Implicit powers are those powers that are implied in the operation of the Executive Branch, either as a result of the President's explicit powers or from the fact that the President is the head of the Executive Branch of the United States government.

\subsection{Explicit Powers}
The Constitution grants the President a number of explicit powers, including those detailed below.


\subsubsection{Veto Power}
The President has the power to veto any law passed by Congress, subject to being overriden by a two-thirds vote of Congress.\footnote{Article I, Section 7  (``Every bill which shall have passed the House of Representatives and the Senate, shall, before it become a law, be presented to the President of the United States; if he approve he shall sign it, but if not he shall return it, with his objections to that House in which it shall have originated, who shall enter the objections at large on their journal, and proceed to reconsider it. If after such reconsideration two thirds of that House shall agree to pass the bill, it shall be sent, together with the objections, to the other House, by which it shall likewise be reconsidered, and if approved by two thirds of that House, it shall become a law.'').}

\subsubsection{Commander-in-Chief}
The President is the Commander in Chief of the United States military, as well as of the Militia of the States when called into federal service.\footnote{Article II, Section 1 (``The President shall be commander in chief of the Army and Navy of the United States, and of the militia of the several states, when called into the actual service of the United States'').}
The Militia of the States consists of male United States citizens between the ages of 17 and 45, and female United States citizens who are members of the National Guard.\footnote{10 USC \S\ 311 (``(a) The militia of the United States consists of all able-bodied males at least 17 years of age and, except as provided in section 313 of title 32, under 45 years of age who are, or who have made a declaration of intention to become, citizens of the United States and of female citizens of the United States who are members of the National Guard.  (b) The classes of the militia are— (1) the organized militia, which consists of the National Guard and the Naval Militia; and (2) the unorganized militia, which consists of the members of the militia who are not members of the National Guard or the Naval Militia.'').}

\subsubsection{Pardon}
The President has the power to grant pardons or reprieves for crimes.\footnote{Article II, Section 2 (``he shall have power to grant reprieves and pardons'').}  However, the President can only grant pardons for federal crimes (i.e. the President cannot grant a pardon for a State crime) and the President cannot grant a pardon in cases of impeachment by Congress.\footnote{Article II, Section 2 (``he shall have power to grant reprieves and pardons for offenses against the United States, except in cases of impeachment'').}

\subsubsection{Foreign Policy}
The President is the United States' chief diplomat and the head of the United States' foreign policy.
Accordingly, the Constitution assigns the President a number of powers to carry out this function.

\begin{itemize}

\item \textbf{Treaties} The Constitution assigns the President the power to make treaties with foreign nations, provided that the Senate approves the treaties by a two-thirds vote.\footnote{Article II, Section 2 (``He shall have power, by and with the advice and consent of the Senate, to make treaties, provided two thirds of the Senators present concur'').}

\item \textbf{Receiving Ambassadors and Diplomats} As the United States' chief diplomat, the Constitution assigns the President the duty to receive all diplomats and ambassadors from foreign nations.\footnote{Article II, Section 3 (``he shall receive ambassadors and other public ministers'').}

\item \textbf{Appointing Diplomats to Other Nations}
The Constitution gives the President the power to appoint the United States' ambassadors, public ministers, and consuls, provided the Senate approves such appointments.\footnote{Article II, Section 2 (``and he shall nominate, and by and with the advice and consent of the Senate, shall appoint ambassadors, other public ministers and consuls'').}

\end{itemize}

\subsubsection{Officers of the United States}
The Constitution and federal law give the President the power to appoint various domestic officers of the United States government.

\begin{itemize}

\item \textbf{Secretaries}
The Constitution gives the President the power to appoint, with the approval of the Senate, the top officers of the United States government.\footnote{Article II, Section 2 (``He shall have power, by and with the advice and consent of the Senate … all other officers of the United States, whose appointments are not herein otherwise provided for, and which shall be established by law'').}
This includes the heads of the various executive departments, such as the Secretary of Defense, the Secretary of State, etc.

\item \textbf{Supreme Court Justices}
The Constitution assigns to the President the power to appoint, with the approval of the Senate, the Justices of the Supreme Court.\footnote{Article II, Section 2 (``he shall nominate, and by and with the advice and consent of the Senate … judges of the Supreme Court.'').}

\item \textbf{Judges of Inferior Courts}
Federal law assigns to the President the power to appoint, with the approval of the Senate, judges of the lower federal courts (i.e. courts below the Supreme Court).\footnote{28 USC \S\ 44(a) (``The President shall appoint, by and with the advice and consent of the Senate, circuit judges for the several circuits''); 28 USC \S\ 133(a) (``The President shall appoint, by and with the advice and consent of the Senate, district judges for the several judicial districts'')}

\item \textbf{Lower Officers}
Because making the President obtain Senate approval for every low-level clerk in every department of the Executive Branch would be tedious and inefficient, the Constitution provides that Congress can by law give to the President the power to appoint or hire members of the Executive Branch without the need for any Congressional approval.\footnote{Article II, Section 2 (``Congress may by law vest the appointment of such inferior officers, as they think proper, in the President alone, in the courts of law, or in the heads of departments'').}  Congress has done this, generally, in Title 3 of the United States Code.\footnote{See, for example 3 U.S.C. \S\ 301 and 302; 3 U.S.C. \S\ 301 (``The President of the United States is authorized to designate and empower the head of any department or agency in the executive branch, or any official thereof who is required to be appointed by and with the advice and consent of the Senate, to perform without approval, ratification, or other action by the President (1) any function which is vested in the President by law, or (2) any function which such officer is required or authorized by law to perform only with or subject to the approval, ratification, or other action of the President: Provided, That nothing contained herein shall relieve the President of his responsibility in office for the acts of any such head or other official designated by him to perform such functions. Such designation and authorization shall be in writing, shall be published in the Federal Register, shall be subject to such terms, conditions, and limitations as the President may deem advisable, and shall be revocable at any time by the President in whole or in part.'').}

\item \textbf{Recess appointments}
As described above, a number of the President's appointment powers require Senate approval.
However, Congress is not in session year round, and Article II, Section 2 provides the President the power to make limited ``recess appointments'' when the Senate is not in session.\footnote{Article II, Section 2 (``The President shall have power to fill up all vacancies that may happen during the recess of the Senate, by granting commissions which shall expire at the end of their next session.'').}  However, recess appointments are temporary and last only until the end of the Senate's next session.

\end{itemize}

\subsubsection{Convening and Adjourning Congress}
Under normal circumstances, Congress controls when it meets and adjourns.  
However, when Congress cannot agree on when to adjourn, the President has the power to adjourn Congress ``to such time as he shall think proper.''\footnote{Article II, Section 3 (``in case of disagreement between them, with respect to the time of adjournment, he may adjourn them to such time as he shall think proper'').}
Further, if Congress is not in session and the need arises, the President has the power to convene Congress for an emergency session.\footnote{Article II, Section 3 (``he may, on extraordinary occasions, convene both Houses, or either of them,'').}

\subsection{Implied Powers, Generally}
In addition to the President's express powers that are explicitly granted by either the Constitution or federal law, the President also possesses implied powers as well.  Implied powers are those powers that are ``reasonably appropriate and relevant to the exercise of an express power'' such that they are considered to implicitly accompany the grant of an express power; that is, the President's implied powers are those powers implied by logical necessity from express grants of power.\footnote{\textit{United States v. Nixon}, 418 US 683, 705 FN16 (1974) (``The rule of constitutional interpretation announced in \textit{McCulloch v. Maryland}, 4 Wheat. 316, that that which was reasonably appropriate and relevant to the exercise of a granted power was to be considered as accompanying the grant, has been so universally applied that it suffices merely to state it.'').}


While there has been some challenge to the existence of non-textual powers (i.e. powers not expressly granted to the President via either the Constitution or Federal law), the Supreme Court has expressed the opinion on several occasions that the existence of the President's implied powers is to be considered axiomatic.\footnote{\textit{United States v. Nixon}, 418 US 683, 705 FN16 (1974) (``The rule of constitutional interpretation announced in \textit{McCulloch v. Maryland}, 4 Wheat. 316, that that which was reasonably appropriate and relevant to the exercise of a granted power was to be considered as accompanying the grant, has been so universally applied that it suffices merely to state it.'').}

However, while the \textit{existence} of the President's implied powers is considered axiomatic, the exact \textit{boundaries} of those powers are not always clearly defined.

To this end, the Supreme Court has developed a framework to determine whether an act by the President exceeds the limits of the Executive's power.
The Court always begins with the integral refrain that the President's authority to act must always stem either from an act of Congress or from the Constitution itself.\footnote{\textit{Medellín v. Texas}, 552 US 491, 128 S. Ct. 1346, 1368 (2008)(``The President's authority to act, as with the exercise of any governmental power, `must stem either from an act of Congress or from the Constitution itself.''')}
With this in mind, the Court then determines which of three categories the challenged Executive action falls into, and evaluates the validity of the President's exercise of power within the context of that individual category.\footnote{\textit{Medellín v. Texas}, 552 US 491, 128 S. Ct. 1346, 1368 (2008)(``Justice Jackson's familiar tripartite scheme provides the accepted framework for evaluating executive action in this area.'')}  


\begin{itemize}
\item \textbf{Expressly Authorized By Congress} According to the Supreme Court, when the President acts pursuant to an express or implied authorization by Congress, then his authority is at its maximum, ``for it includes all that he possesses in his own right plus all that Congress can delegate.''\footnote{\textit{Youngstown Sheet \& Tube Co. v. Sawyer}, 343 US 579, 635 -- 37 (1952)(``1. When the President acts pursuant to an express or implied authorization of Congress, his authority is at its maximum, for it includes all that he possesses in his own right plus all that Congress can delegate.  In these circumstances, and in these only, may he be said (for what it may be worth) to personify the federal sovereignty. If his act is held unconstitutional under these circumstances, it usually means that the Federal Government as an undivided whole lacks power. A seizure executed by the President pursuant to an Act of Congress would be supported by the strongest of presumptions and the widest latitude of judicial interpretation, and the burden of persuasion would rest heavily upon any who might attack it.''); \textit{Medellín v. Texas}, 552 US 491, 128 S. Ct. 1346, 1368 (2008)(``First, "[w]hen the President acts pursuant to an express or implied authorization of Congress, his authority is at its maximum, for it includes all that he possesses in his own right plus all that Congress can delegate."'').}  Backed by the authorization of Congress, the President stands in the strongest position to defend the validity of his actions.  
An act by the President pursuant to an act of Congress ``is supported by the strongest of presumptions and the widest latitude of judicial interpretation, and the burden of persuasion rests heavily upon any who might attack it.''\footnote{\textit{Youngstown Sheet \& Tube Co. v. Sawyer}, 343 US 579, 635 -- 37 (1952)(``A seizure executed by the President pursuant to an Act of Congress would be supported by the strongest of presumptions and the widest latitude of judicial interpretation, and the burden of persuasion would rest heavily upon any who might attack it.'').}
If, under these circumstances, the President's act is held to be unconstitutional, ``it usually means that the Federal Government as an undivided whole lacks power'' to effectuate such an act.\footnote{\textit{Youngstown Sheet \& Tube Co. v. Sawyer}, 343 US 579, 635 -- 37 (1952)(``If his act is held unconstitutional under these circumstances, it usually means that the Federal Government as an undivided whole lacks power.'').}

\item \textbf{In the Absence of Congressional Action} When the President acts without either a congressional grant or denial of authority, then the Executive's actions can only rely upon the President's own independent powers to validate their legitimacy.\footnote{\textit{Youngstown Sheet \& Tube Co. v. Sawyer}, 343 US 579, 637 (1952)(``When the President acts in absence of either a congressional grant or denial of authority, he can only rely upon his own independent powers''); \textit{Medellín v. Texas}, 552 US 491, 128 S. Ct. 1346, 1368 (2008)(``Second, "[w]hen the President acts in absence of either a congressional grant or denial of authority, he can only rely upon his own independent powers"'').}
However, in situations where Congress and the President potentially have overlapping authority or where the distribution of power is uncertain, Congressional inertia, indifference, or inaction can actually \textit{bolster} the validity the President's acts, because, as a practical matter, Congress' silence may ``enable, if not invite'' independent presidential action.\footnote{\textit{Youngstown Sheet \& Tube Co. v. Sawyer}, 343 US 579, 637 (1952)(``When the President acts in absence of either a congressional grant or denial of authority, he can only rely upon his own independent powers, but there is a zone of twilight in which he and Congress may have concurrent authority, or in which its distribution is uncertain. Therefore, congressional inertia, indifference or quiescence may sometimes, at least as a practical matter, enable, if not invite, measures on independent presidential responsibility. In this area, any actual test of power is likely to depend on the imperatives of events and contemporary imponderables rather than on abstract theories of law.''); \textit{Medellín v. Texas}, 552 US 491, 128 S. Ct. 1346, 1368 (2008)(``Second, "[w]hen the President acts in absence of either a congressional grant or denial of authority, he can only rely upon his own independent powers, but there is a zone of twilight in which he and Congress may have concurrent authority, or in which its distribution is uncertain." Id., at 637, 72 S.Ct. 863. In this circumstance, Presidential authority can derive support from "congressional inertia, indifference or quiescence." Ibid.'').}
Under these circumstances, a determination of the President's authority will likely depend on the circumstances of the case and the ``imperatives of events.''\footnote{\textit{Youngstown Sheet \& Tube Co. v. Sawyer}, 343 US 579, 637 (1952)(``In this area, any actual test of power is likely to depend on the imperatives of events and contemporary imponderables rather than on abstract theories of law.'').}

\item \textbf{Against the Will of Congress} When the President acts in direct contravention to the expressed or implied will of Congress, his power is at its lowest ebb.  Under such circumstances his actions can only be sustained by the Judicial Branch by disabling the Legislative Branch from acting upon the subject -- i.e. by removing from Congress the authority to legislate or otherwise take action on the subject.\footnote{\textit{Youngstown Sheet \& Tube Co. v. Sawyer}, 343 US 579, 637 -- 38 (1952)(``When the President takes measures incompatible with the expressed or implied will of Congress, his power is at its lowest ebb, for then he can rely only upon his own constitutional powers minus any constitutional powers of Congress over the matter. Courts can sustain exclusive presidential control in such a case only by disabling the Congress from acting upon the subject. Presidential claim to a power at once so conclusive and preclusive must be scrutinized with caution, for what is at stake is the equilibrium established by our constitutional system.''); \textit{Medellín v. Texas}, 552 US 491, 128 S. Ct. 1346, 1368 (2008)(``Finally, "[w]hen the President takes measures incompatible with the expressed or implied will of Congress, his power is at its lowest ebb," and the Court can sustain his actions "only by disabling the Congress from acting upon the subject."'').}  As a practical matter, this means that the Court will have to find that Congress does not have the power to restrict the President from taking such actions

\end{itemize}

\subsection{Executive Privilege}

One of the generally-accepted implied powers of the President is the ``Executive Privilege.''  This is the President's exemption from compulsion by another branch of the government to reveal what he deems to be private communications.  Typically this arises when Congress or a Court issues a subpoena requiring a member of the Executive Branch to either testify under oath or to produce documentary evidence.  For example, suppose Congress is conducting an investigation into the Department of Justice, and issues a subpoena requiring the Attorney General to appear and give sworn testimony before Congress.  If the President doesn't want the Attorney General to testify in front of Congress, the President can assert Executive Privilege and refuse to allow the Attorney General to comply with the subpoena.  

While this may aggravate the branch issuing the subpoena, once the President asserts Executive Privilege there is little, practically speaking, that the issuing branch can do to enforce the subpoena.  Typically when a subpoenaed individual refuses to comply with a subpoena, the issuing branch submits the matter to the Executive Branch for enforcement.  However, when the President asserts Executive Privilege, it is the Executive Branch itself that is resisting the subpoena.  Thus, the issuing branch is left with little recourse because its normal method of enforcement (the Executive Branch) is the party resisting the subpoena and the issuing branch has no mechanism to force the Executive Branch to comply with the subpoena.

\subsubsection{Basis of Privilege}
There are several bases for the Executive Privilege, including (1) the practical need for confidentiality and (2) the separation of powers.

As to the first basis, the Supreme Court describes the practical need for confidentiality as ``the valid need for protection of communications between high Government officials and those who advise and assist them in the performance of their manifold duties.''\footnote{\textit{United States v. Nixon}, 418 US 683, 705 (1974)(``The first ground is the valid need for protection of communications between high Government officials and those who advise and assist them in the performance of their manifold duties'')}
The Supreme Court has said that the importance of this confidentiality ``is too plain to require further discussion,'' that ``human experience teaches that those who expect public dissemination of their remarks may well temper candor with a concern for appearances and for their own interests to the detriment of the decisionmaking process.''\footnote{\textit{United States v. Nixon}, 418 US 683, 705 (1974)(``the importance of this confidentiality is too plain to require further discussion. Human experience teaches that those who expect public dissemination of their remarks may well temper candor with a concern for appearances and for their own interests to the detriment of the decisionmaking process.'')}
The concept is that the lack of protection will chill frankness and candor to the President from his advisers if they think that their remarks will later be publicly disseminated and picked over by the public and political enemies.  Under these circumstances, the President's advisors may withhold otherwise important comments or ideas that would otherwise be critical in advising the President.

Second, as to the separation of powers, the Supreme Court considers the Executive Privilege to derive from the supremacy of each branch within its own assigned area of constitutional duties.\footnote{\textit{United States v. Nixon}, 418 US 683, 705 (1974)(``Whatever the nature of the privilege of confidentiality of Presidential communications in the exercise of Art. II powers, the privilege can be said to derive from the supremacy of each branch within its own assigned area of constitutional duties.'').}
Just as implicit powers and privileges of each branch come from the explicit enumerated powers of that branch, ``it is accepted that the protection of the confidentiality of Presidential communications has similar constitutional underpinnings.''\footnote{\textit{United States v. Nixon}, 418 US 683, 705 -- 06 (1974)(``Certain powers and privileges flow from the nature of enumerated powers; the protection of the confidentiality of Presidential communications has similar constitutional underpinnings.'').}

\subsubsection{Limitations}
Because Executive Privilege is based on (1) the President's need for confidentiality and (2) the principle of separation of powers, an assertion of Executive Privilege is usually afforded high deference by the other branches of the government.\footnote{\textit{United States v. Nixon}, 418 US 683, 706 (1974)(``The President's need for complete candor and objectivity from advisers calls for great deference from the courts'').}
However, Executive Privilege is not absolute, and a claim of Privilege by the President must be considered in light of other important fundamental interests, such as the our country's commitment to the rule of law.\footnote{\textit{United States v. Nixon}, 418 US 683, 706 (1974)(``However, neither the doctrine of separation of powers, nor the need for confidentiality of high-level communications, without more, can sustain an absolute, unqualified Presidential privilege of immunity from judicial process under all circumstances ... when the privilege depends solely on the broad, undifferentiated claim of public interest in the confidentiality of such conversations, a confrontation with other values arises.''); \textit{United States v. Nixon}, 418 US 683, 708 (1974)(``this presumptive privilege must be considered in light of our historic commitment to the rule of law'').}


Therefore, whenever a party seeking information from the Executive raises a challenge to a claim of Privilege by the President, a reviewing Court must balance the Privilege's underlying bases against the competing interest of the party challenging the Privilege.

Take, for example, a President's assertion of Privilege against a Court-issued subpoena in an ongoing criminal prosecution.  A reviewing Court will compare the need for the subpoenaed evidence against the (1) the Executive Branch's need for confidentiality and (2) the Constitutional principle of separation of powers.  As to the first basis, while the Court will respect a President's legitimate need for confidentiality, when a President's claim of Privilege is based only on a \textit{generalized} need for confidentiality and not a particularized necessity for military or diplomatic secrecy, then the Privilege must yield to a demonstrated, specific need for evidence in a pending criminal trial.\footnote{\textit{United States v. Nixon}, 418 US 683, 712 (1974)(``The generalized assertion of privilege must yield to the demonstrated, specific need for evidence in a pending criminal trial.'').}  

As to the second basis, while Courts will respect the principle of separation of powers, they will not read into such principle complete autonomy for the Executive Branch.  The Constitution does not envision absolute separation of the three branches, but rather interdependence and reciprocity.\footnote{\textit{United States v. Nixon}, 418 US 683, 707 (1974)(``While the Constitution diffuses power the better to secure liberty, it also contemplates that practice will integrate the dispersed powers into a workable government. It enjoins upon its branches separateness but interdependence, autonomy but reciprocity.'').}  Thus, when the President has not demonstrated an impairment to the functioning of the Executive Branch, a generalized claim of separation of powers will not defeat a validly-issued subpoena that is essential to the enforcement of criminal statutes.  To do otherwise would, according to the Supreme Court, upset the constitutional balance of a workable government and gravely impair the role of the different branches under the Constitution.\footnote{\textit{United States v. Nixon}, 418 US 683, 707 (1974)(``To read the Art. II powers of the President as providing an absolute privilege as against a subpoena essential to enforcement of criminal statutes on no more than a generalized claim of the public interest in confidentiality of nonmilitary and nondiplomatic discussions would upset the constitutional balance of "a workable government" and gravely impair the role of the courts under Art. III'').}


\subsection{War Powers}
As the Commander in Chief of the United States' military forces, the President is the central figure in the supervision and execution of any United States military action.   Concomitant with this explicit power is the President's implied power to wage war successfully.\footnote{\textit{Hamdi v. Rumsfeld}, 542 U. S. 507, 536 (2004)(``The war power "is a power to wage war successfully, and thus it permits the harnessing of the entire energies of the people in a supreme cooperative effort to preserve the nation."'').}
This implied power is the President's ``war power'' and it ``permits the harnessing of the entire energies of the people in a supreme cooperative effort to preserve the nation.''\footnote{\textit{Hamdi v. Rumsfeld}, 542 U. S. 507, 536 (2004)(``The war power "is a power to wage war successfully, and thus it permits the harnessing of the entire energies of the people in a supreme cooperative effort to preserve the nation.'').}

However, while the necessity of successfully prosecuting a military action lends compelling support to a President's assertion of constitutional authority in pursuit of such an objective, the President is limited in how far he can legitimately stretch the bounds of his implied war powers.

At the very least, basic constitutional structures must remain intact, even during times of military conflict; a state of war is not a blank check for the President.\footnote{\textit{Hamdi v. Rumsfeld}, 542 U. S. 507, 536 (2004)(``We have long since made clear that a state of war is not a blank check for the President when it comes to the rights of the Nation's citizens … even the war power does not remove constitutional limitations safeguarding essential liberties.'').}  
The President cannot use his war powers to run roughshod over the essential liberties and rights of the United States' citizens.\footnote{\textit{Hamdi v. Rumsfeld}, 542 U. S. 507, 536 (2004)(``We have long since made clear that a state of war is not a blank check for the President when it comes to the rights of the Nation's citizens … even the war power does not remove constitutional limitations safeguarding essential liberties.'').}  
Further, the President cannot use his war powers to contravene the essential structure of the United States government established by the Constitution.  Even during the exigencies of war the President cannot use his war powers to become omnipotent within the federal government or subvert the separation of powers in the Constitution.\footnote{\textit{Hamdi v. Rumsfeld}, 542 U. S. 507, 536 (2004)(``Whatever power the United States Constitution envisions for the Executive in its exchanges with other nations or with enemy organizations in times of conflict, it most assuredly envisions a role for all three branches when individual liberties are at stake … it was "the central judgment of the Framers of the Constitution that, within our political scheme, the separation of governmental powers into three coordinate Branches is essential to the preservation of liberty"''); \textit{Hamdan v. Rumsfeld}, 548 US 557, 591 -- 92 (2006)(``The power to make the necessary laws is in Congress; the power to execute in the President. Both powers imply many subordinate and auxiliary powers. Each includes all authorities essential to its due exercise. But neither can the President, in war more than in peace, intrude upon the proper authority of Congress, nor Congress upon the proper authority of the President''); The Federalist No. 47 (``The accumulation of all powers legislative, executive and judiciary in the same hands … may justly be pronounced the very definition of tyranny'').}


The Supreme Court has addressed asserted implied Presidential power during war time on a number of occasions.
During the Civil War, the Court found that the Executive Branch could not try a civilian in a military tribunal in a State that was not in insurrection.\footnote{\textit{Ex parte Milligan}, 4 Wall. 2, 121 -- 22 (1866) (``But it is said that the jurisdiction is complete under the "laws and usages of war."  It can serve no useful purpose to inquire what those laws and usages are, whence they originated, where found, and on whom they operate; they can never be applied to citizens in states which have upheld the authority of the government, and where the courts are open and their process unobstructed. This court has judicial knowledge that in Indiana the Federal authority was always unopposed, and its courts always open to hear criminal accusations and redress grievances; and no usage of war could sanction a military trial there for any offence whatever of a citizen in civil life, in nowise connected with the military service'').}
During the Korean War, the Court found that the President, lacking any Congressional backing or statutory authority, could not rely on his implied war powers to seize and operate privately owned United States steel mills even for the purpose of supplying troops in battle with necessary supplies and equipment.\footnote{\textit{Youngstown Sheet \& Tube Co. v. Sawyer}, 343 U. S. 579, 588 (1952)(``The President's order does not direct that a congressional policy be executed in a manner prescribed by Congress —it directs that a presidential policy be executed in a manner prescribed by the President … Congress … can authorize the taking of private property for public use. It can make laws regulating the relationships between employers and employees, prescribing rules designed to settle labor disputes, and fixing wages and working conditions in certain fields of our economy. The Constitution does not subject this lawmaking power of Congress to presidential or military supervision or control.'').}
In the first decade of the 21st Century during the ``War on Terror,'' the Supreme Court rebuked the President's claim that the Executive Branch could indefinitely detain a United States Citizen as an ``enemy combatant'' without affording that citizen basic due process rights to challenge his classification as an enemy combatant.\footnote{\textit{Hamdi v. Rumsfeld}, 542 U. S. 507 (2004).}

\section{Removal from Office}

The Constitution provides for the removal of the President from office, voluntarily or involuntarily, temporarily or permanently, during his four-year term.

\subsection{Temporary Removal}

\subsubsection{Voluntary Temporary Recusal}
If the President finds that he is temporarily unable to continue acting as President, he may, pursuant to Section 3 of the 25th Amendment, voluntarily recuse himself from the office of the Presidency for a temporary period.  To do so, the President must send a letter to the Speaker of the House and the President Pro Temp of the Senate stating that he is temporarily unable to continue acting as President.\footnote{Amendment 25, Section 3 (``Whenever the President transmits to the President pro tempore of the Senate and the Speaker of the House of Representatives his written declaration that he is unable to discharge the powers and duties of his office, and until he transmits to them a written declaration to the contrary, such powers and duties shall be discharged by the Vice President as Acting President.'').}
This recusal lasts until the President sends a second letter revoking the original letter.\footnote{Amendment 25, Section 3 (``Whenever the President transmits to the President pro tempore of the Senate and the Speaker of the House of Representatives his written declaration that he is unable to discharge the powers and duties of his office, and until he transmits to them a written declaration to the contrary, such powers and duties shall be discharged by the Vice President as Acting President.'').}  During the recusal period, the Vice President serves as the acting President.\footnote{Amendment 25, Section 3 (``until he transmits to them a written declaration to the contrary, such powers and duties shall be discharged by the Vice President as Acting President.'').}

\subsubsection{Involuntary Temporary Removal}
Section 4 of the 25th Amendment provides a method for involuntarily removing the President for a temporary period of time.
If the Vice President and a majority of either the Cabinet or another group designated by Congress determine that the President is unable to continue acting as the President, they can send a letter to the Speaker of the House and the President Pro Temp of the Senate stating the same.\footnote{Amendment 25, Section 4 (``Whenever the Vice President and a majority of either the principal officers of the executive departments or of such other body as Congress may by law provide, transmit to the President pro tempore of the Senate and the Speaker of the House of Representatives their written declaration that the President is unable to discharge the powers and duties of his office, the Vice President shall immediately assume the powers and duties of the office as Acting President.'').}
Upon transmission of this letter, the President is temporarily relieved of office and the Vice President immediately becomes the acting President.

To return to office, the President must send a letter to the Speaker of the House and President Pro Temp of the Senate stating that he is capable of acting as President.\footnote{Amendment 25, Section 4 (``Thereafter, when the President transmits to the President pro tempore of the Senate and the Speaker of the House of Representatives his written declaration that no inability exists, he shall resume the powers and duties of his office'').}
Upon transmission of this letter, the President immediately resumes the office of the Presidency and remains in office, unless the Vice President and a majority of the same group as above transmit a letter within four days stating that the President is unable to continue acting as President.\footnote{Amendment 25, Section 4 (``Thereafter, when the President transmits to the President pro tempore of the Senate and the Speaker of the House of Representatives his written declaration that no inability exists, he shall resume the powers and duties of his office unless the Vice President and a majority of either the principal officers of the executive department or of such other body as Congress may by law provide, transmit within four days to the President pro tempore of the Senate and the Speaker of the House of Representatives their written declaration that the President is unable to discharge the powers and duties of his office.'').}

At that point, the Constitution delegates the responsibility to Congress to decide the issue.  If Congress decides by a two-thirds vote of both houses that the President is unable to continue acting as President, then the Vice President continues as acting President.  Otherwise, the President resumes the office of the Presidency.\footnote{Amendment 25, Section 4 (``Thereupon Congress shall decide the issue, assembling within forty-eight hours for that purpose if not in session. If the Congress, within twenty-one days after receipt of the latter written declaration, or, if Congress is not in session, within twenty-one days after Congress is required to assemble, determines by two-thirds vote of both Houses that the President is unable to discharge the powers and duties of his office, the Vice President shall continue to discharge the same as Acting President; otherwise, the President shall resume the powers and duties of his office.'').}

\subsection{Voluntary and Involuntary Permanent Removal}

\subsubsection{Death or Resignation}
The Constitution provides that upon the death or resignation of the President, the Vice President becomes President.\footnote{Amendment 25, Section 1 (``In case of the removal of the President from office or of his death or resignation, the Vice President shall become President.'')}

\subsubsection{Impeachment}

The Constitution also provides for the involuntary permanent removal of the President through the impeachment process.  Congress can impeach and convict (and thus remove from office) the President on the basis of (1) treason, (2) bribery, or (3) other high crimes and misdemeanors.\footnote{Article II, Section 4 (``The President, Vice President and all civil officers of the United States, shall be removed from office on impeachment for, and conviction of, treason, bribery, or other high crimes and misdemeanors.'').}
As to the first ground, the Constitution defines treason as levying war against the United States, or in adhering or giving aid and comfort to the enemies of the United States.\footnote{Article III, Section 3 (``Treason against the United States, shall consist only in levying War against them, or in adhering to their Enemies, giving them Aid and Comfort. No Person shall be convicted of Treason unless on the Testimony of two Witnesses to the same overt Act, or on Confession in open Court.'').}  As to the second ground, bribery is a well known offense and is susceptible of easy definition.

However, as to the third ground, the meaning of the phrase ``high crimes and misdemeanors'' is not as easily ascertained.  In the Federalist \#65, Alexander Hamilton says that the impeachment and conviction process is, for all intents and purposes, a \textit{political} one, and that the offenses meant to be covered by impeachment trials are those that violate the public trust:

\begin{quote}
The subjects of its jurisdiction are those offenses which proceed from the misconduct of public men, or, in other words, from the abuse or violation of some public trust. They are of a nature which may with peculiar propriety be denominated POLITICAL, as they relate chiefly to injuries done immediately to the society itself.\footnote{Federalist \#65.}
\end{quote}

Because the impeachment and trial of the President is a power delegated to Congress by Article I of the Constitution, this leaves to Congress a wide discretion as to what falls within this definition (and by implication a wide discretion as to what Congress can remove a President from office for).  Because this determination is, as Hamilton says, at its heart a \textit{political} one, ultimately it means, more than likely, whatever Congress says it means.

\section{Presidential Succession}
So what happens when there is a vacancy in the office of the Presidency?
The Constitution lays out the order of presidential succession.

\subsection{If President Dies After Elected But Before Taking Office}
If the President dies after elected but before taking office, Section 3 of the 20th Amendment provides that the Vice President elect will take office as President.\footnote{Amendment 20, Section 3 (``If, at the time fixed for the beginning of the term of the President, the President elect shall have died, the Vice President elect shall become President.'').}

\subsection{If President Dies, Resigns, or is Removed}
If the President dies, resigns, or is removed from office, then the Constitution provides that the Vice President will take over as President.\footnote{Amendment 25, Section 1 (``In case of the removal of the President from office or of his death or resignation, the Vice President shall become President.'').}
Further, Article II, Section 1 of the Constitution provides that Congress can set the order of presidential succession by law for the cases where, either by death, resignation, or removal from office, neither the President nor the Vice President are able to hold the office of the President.\footnote{Article II, Section 1 (``In case of the removal of the President from office, or of his death, resignation, or inability to discharge the powers and duties of the said office, the same shall devolve on the Vice President, and the Congress may by law provide for the case of removal, death, resignation or inability, both of the President and Vice President, declaring what officer shall then act as President, and such officer shall act accordingly, until the disability be removed, or a President shall be elected.'')}

Per 3 U.S.C. \S\ 19, Congress has set the order of presidential succession after the Vice President as follows\footnote{3 U.S.C. \S\ 19(a) (``(1) If, by reason of death, resignation, removal from office, inability, or failure to qualify, there is neither a President nor Vice President to discharge the powers and duties of the office of President, then the Speaker of the House of Representatives shall, upon his resignation as Speaker and as Representative in Congress, act as President.  (2) The same rule shall apply in the case of the death, resignation, removal from office, or inability of an individual acting as President under this subsection.''); 3 U.S.C. \S\ 19(b) (''If, at the time when under subsection (a) of this section a Speaker is to begin the discharge of the powers and duties of the office of President, there is no Speaker, or the Speaker fails to qualify as Acting President, then the President pro tempore of the Senate shall, upon his resignation as President pro tempore and as Senator, act as President.''); 3 U.S.C. \S\ 19(d) (``(1) If, by reason of death, resignation, removal from office, inability, or failure to qualify, there is no President pro tempore to act as President under subsection (b) of this section, then the officer of the United States who is highest on the following list, and who is not under disability to discharge the powers and duties of the office of President shall act as President: Secretary of State, Secretary of the Treasury, Secretary of Defense, Attorney General, Secretary of the Interior, Secretary of Agriculture, Secretary of Commerce, Secretary of Labor, Secretary of Health and Human Services, Secretary of Housing and Urban Development, Secretary of Transportation, Secretary of Energy, Secretary of Education, Secretary of Veterans Affairs, Secretary of Homeland Security.'').}:
\begin{enumerate}
\item Speaker of the House
\item Senate Pro Tempore
\item Secretary of State
\item Secretary of the Treasury
\item Secretary of Defense
\item Attorney General
\item Secretary of the Interior
\item Secretary of Agriculture
\item Secretary of Commerce
\item Secretary of Labor
\item Secretary of Health and Human Services
\item Secretary of Housing and Urban Development
\item Secretary of Transportation
\item Secretary of Energy
\item Secretary of Education
\item Secretary of Veterans Affairs
\item Secretary of Homeland Security
\end{enumerate}

\section{Vacancy in Vice-Presidency}
Per the 25th Amendment, when there is a vacancy in the office of the Vice President, the President can nominate a Vice President, who will assume the office upon confirmation by a majority vote by both houses of Congress.\footnote{Amendment 25, Section 2 (``Whenever there is a vacancy in the office of the Vice President, the President shall nominate a Vice President who shall take office upon confirmation by a majority vote of both Houses of Congress.'').}


