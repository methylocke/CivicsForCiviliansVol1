% ******************************************************** %
% Copyright © Jonathan Gitlin 2011 – 2013
%
% This work is licensed under the Creative Commons 
% Attribution-NonCommercial-ShareAlike 3.0 Unported License
%
% For More Information See the Creative Commons Website
% http://creativecommons.org
%
% For A human readable summary, See:
% http://creativecommons.org/licenses/by-nc-sa/3.0/
%
% For the Full Text of the License:
% http://creativecommons.org/licenses/by-nc-sa/3.0/legalcode
%
% No Claim Is Made To Original Works 
% Of Other Sources Cited Within This Work
%
% ******************************************************** %

\chapter{Introduction}

\section{Purpose of Chapter}
The purpose of this Chapter is to introduce the reader to the basic concepts and ideas of government, as well as to provide a brief survey of some of the more historically important philosophical views on the subject that have influenced the development and structure of the United States government.

\section{What is Government}
The first and most simple question is -- what is government?  Succinctly, a government is the machine by which society exercises its collective power to control the behavior of society or to do things for the benefit of society.  In its simplest form, a government is the machine by which society uses its collective power to exercise its will on society.

\section{Why do we have government?}
So, why do we have government?  Many philosophers have postulated different reasons for instituting and maintaining a government.  There are, however, some philosophers whose ideas have influenced the structure and ideology of government in the United States more than others.

\subsection{Rousseau}
Jean-Jacques Rousseau was an 18th Century philosopher whose political philosophy permeates many aspects of modern political thought.  In his book \textit{The Social Contract}, he addresses the concept of why men have instituted governments.  Succinctly, Rousseau says that we join ourselves together for the preservation and betterment of all but, in exchange, we submit ourselves to the judgment and decisions of the community.
In Chapter 6 of Book I, entitled ``The Social Pact,'' Rousseau says:

\begin{quote}

I assume that men reach a point where the obstacles to their preservation in a state of nature prove greater than the strength that each man has to preserve himself in that state.  Beyond this point, the primitive condition cannot endure for then the human race will perish if it does not change its mode of existence ... the only way in which they can preserve themselves is by uniting their separate powers in a combination strong enough to overcome any resistance, uniting them so that their powers are directed by a single motive and act in concert.  Such a sum of forces can be produced only by the union of separate men ... If, then, we eliminate from the social pact everything that is not essential to it, we find it comes down to this: `Each one of us puts into the community his person and all his powers under the supreme direction of the general will; and as a body, we incorporate every member as an indivisible part of the whole.'

\end{quote}

He goes on in Chapter 5 of Book II, entitled ``The Right of Life and Death'':

\begin{quote}
The purpose of the social treaty is the preservation of the contracting parties.  Whoever wills the end wills also the means, and certain risks, even certain casualties are inseparable from these means.
\end{quote}

\subsection{John Locke}
John Locke was a 17th Century British philosopher whose ideas and thinking on government form the basis of many of the concepts found both in the Declaration of Independence and Constitution of the United States.  His most influential work is, arguably, his \textit{Two Treatises on Government}, which outlines his ideas on the structure and purpose of governments.  In his \textit{Second Treatise on Government}, Locke lists two goals of civil government: (1) a justice system and (2) the preservation of property.

\subsubsection{Justice System}
To explain the benefits of a justice system, Locke goes back to pre-government culture.  He begins with all men in a state of ``nature,'' without government, where each person is left to act as he or she pleases, and to be his or her own judge with respect to (1) that persons own acts and (2) the acts of others.  In an ideal world this independent state of nature would be sufficient for all, because ``all [being] equal and independent, no one ought to harm another in his life, health, liberty, or possessions.''\footnote{Second Treatise on Government, Chapter 2, Section 13.}  However, in reality men often do harm each other in their life, health, liberty, and possessions.  Thus, Locke postulates, we need what is, essentially, a justice system to restrain the ``bad acts'' of men.  Thus, Locke says, 
\begin{quote}
civil government is the proper remedy for the inconveniences of the state of nature, where men [are all] judges in their own case, since it is easy to be imagined, that he who was so unjust as to do his brother an injury, will scarce be so just as to condemn himself for it ... it is unreasonable for men to be judges in their own cases, because self-love will make men partial to themselves and their friends ... [and] ill nature, passion and revenge will carry them too far in punishing others ... [and in this state] nothing but confusion and disorder will follow\footnote{Second Treatise on Government, Chapter 2, Section 13.}
\end{quote}

Thus, according to Locke, we have government to create a justice system because
\begin{enumerate}
\item Men harm each other in their life, health, liberty, and possessions;
\item Self love makes it impossible for men to be proper judges in cases of themselves and their friends; and 
\item Ill nature, passion, and revenge will carry them too far in punishing others.
\end{enumerate}

\subsubsection{Preservation of Property}
Further, in several places Locke also lists the preservation of property as one of the major goals of civil government.

\begin{itemize}
\item In Chapter 9, section 124, of his  \textit{Second Treatise on Government} he says:
``The great and chief end, therefore, of men's uniting into commonwealths, and putting themselves under government, is the preservation of their property.''
\item In Chapter 9, section 127, he says:
[Fleeing the uncertainty and inconveniences of the state of nature, men] ``take sanctuary under the established laws of government, and therein seek the preservation of their property.''
\item In Chapter 11, Section 138, he says:
``the preservation of property [is] the end of government, and that for which men enter into society [for]''
\end{itemize}


\subsection{Adam Smith}
Adam Smith was an 18th Century Scottish moral, political, and economic philosopher whose seminal book, \textit{The Wealth of Nations}, many consider to be the foundation of modern economic theory.  In \textit{The Wealth of Nations}, Adam Smith echoes Locke's sentiments that the preservation of property is the central purpose of civil government.
However, he makes an additional comment regarding this.  In Book 5, Chapter 1, Part 2, titled ``Of the Expense of Justice'' he says
\begin{quote}
Among nations of hunters, as there is scarce any property, or at least none that exceeds the value of two or three days' labour, so there is seldom any established magistrate or any regular administration of justice ... [because] Men who have no property can injure one another only in their persons or reputations ... [and] Men may live together in society with some tolerable degree of security, though there is no civil magistrate to protect them from the injustice of those passions. 
\end{quote}

Smith describes societies of hunters whose members have virtually no property.  According to Smith, in these societies there is often little need for an established justice system because there is little to no property ownership to be protected.  As such, in these societies people can only injure one another's person or reputation.
Accordingly, there is little need for a civil magistrate system because there is little to protect in the society.  

However, Smith says that when people in a society begin to accumulate property then the need for a civil government arises to protect personal property ownership.  Smith goes on to describe, as Locke did, the function of a civil government in protecting property ownership.  

\begin{quote}
But avarice and ambition in the rich, in the poor the hatred of labour and the love of present ease and enjoyment, are the passions which prompt to invade property, passions much more steady in their operation, and much more universal in their influence.  Wherever there is great property there is great inequality. For one very rich man there must be at least five hundred poor, and the affluence of the few supposes the indigence of the many. The affluence of the rich excites the indignation of the poor, who are often both driven by want, and prompted by envy, to invade his possessions. It is only under the shelter of the civil magistrate that the owner of that valuable property, which is acquired by the labour of many years, or perhaps of many successive generations, can sleep a single night in security. He is at all times surrounded by unknown enemies, whom, though he never provoked, he can never appease, and from whose injustice he can be protected only by the powerful arm of the civil magistrate continually held up to chastise it. 
The acquisition of valuable and extensive property, therefore, necessarily requires the establishment of civil government.  
\end{quote}

Succinctly, Smith says that inequality in property ownership among members of a society incites those who have less to want to take from those who have more.  Under such circumstances, it is only with the protection of a civil government that those members of society with a large amount of property can sleep in security.  According to Smith, the rich are ``surrounded by unknown enemies, whom, though he never provoked, he can never appease, and from whose injustice he can be protected only by the powerful arm of the civil magistrate.''  Thus, according to Smith, when there are members of society that have ``valuable and extensive property,'' a civil government is necessary to maintain the ownership of such property.  Smith continues on this tract:

\begin{quote}
Where there is no property, or at least none that exceeds the value of two or three days' labour, civil government is not so necessary.  It is … the inequality of fortune … [that] introduces some degree of that civil government which is indispensably necessary for its own preservation: and it seems to do this naturally, and even independent of the consideration of that necessity ... 
The rich, in particular, are necessarily interested to support that order of things which can alone secure them in the possession of their own advantages.  Men of inferior wealth combine to defend those of superior wealth in the possession of their property, in order that men of superior wealth may combine to defend them in the possession of theirs.  All the inferior shepherds and herdsmen feel that the security of their own herds and flocks depends upon the security of those of the great shepherd or herdsman; that the maintenance of their lesser authority depends upon that of his greater authority, and that upon their subordination to him depends his power of keeping their inferiors in subordination to them ... Civil government, so far as it is instituted for the security of property, is in reality instituted for the defence of the rich against the poor, or of those who have some property against those who have none at all.
\end{quote}

In Smith's description, those with \textit{some} property combine to protect those with \textit{much} property so that those with much wealth will in turn protect those with only some wealth.  Smith continues by saying ``Civil government, so far as it is instituted for the security of property, is in reality instituted for the defence of the rich against the poor, or of those who have some property against those who have none at all.''  Succinctly, Adam Smith says the purpose of civil government is not just the preservation of property, but more specifically, to protect the rich from the poor.

\section{Government In Practice}

Regardless of the philosophical underpinnings and theory, practically speaking we have government today for 2 reasons today: (1) to control the behavior of society and (2) to do things for society.

\subsubsection{Control the Behavior of Society}

First, government is instituted to control the behavior of society.  This includes making laws prohibiting certain types of behavior that society deems harmful.  These laws may be colloquially referred to as ``thou shalt not'' edicts.  But why do we need a government to do this?  Does society need controlling?  In the Federalist \#51, written in the late 18th Century, James Madison said: ``What is government itself, but the greatest of all reflections on human nature? If men were angels, no government would be necessary.''  Madison was stating that we use government to control society because we acknowledge the simple fact that without government humans will, if left unrestrained, commit acts that are harmful to themselves, others, and society as a whole.  Hence, society uses the machine of government to curb behavior that its finds objectionable.

\subsubsection{Provide Services to Foster Society}

Second, modern day governments provide services that foster the health and welfare of society.  Examples of this include various civil services governments provide, such as building and maintaining roads, building and operating fire and police departments, and maintaining county records departments.

These are typically services and activities that we deem so crucial to the welfare of society that we provide them via the machine of government so that they are available to all members of society without the need for multiple individuals or discrete groups of individuals to create and maintain them independently.

\section{Form of Government}

\subsection {Types of Government}
After understanding the what and why of government, the next question to ask is ``what form should government take?''  Societies have developed a number of different types of government over the years.  A small sampling (with a brief description for each) is listed below.

\begin{itemize}
\item \textbf{Monarchy}
In its most basic form, a Monarchy is a government with one person at the head, where all of the powers of the government rest in and emanate from that single person, such as a king or queen. In a Monarchy, the decisions of the government are made and the power of the government is exercised by, either directly or on behalf of, the Monarch.  Examples of this in history include the pre-modern-era governments of England, France, and Spain.

\item \textbf{Democracy}
Contrary to a monarchy, where the power of the government is vested in and emanates from a single person, in a democracy the power of the government is vested in and emanates from the people as a whole.  In a democracy, the people vote, either directly or indirectly, to make the decisions of the society.

\item \textbf{Republic}
In its strictest sense, this is usually accomplished via a representative body, such as a legislature of some kind, wherein the legislators represent the people.\footnote{Examples of this include the United States house of representatives, or, on a smaller scale, most city counsels.}  Unlike a democracy, where the members of society vote directly on a government's affairs, in a Republic, the people have representatives, and the representatives vote on the government's affairs as representatives of the people.  In a republic, the intent is that a smaller group of the members of society shall represent the populace as a whole.  

\item \textbf{Federal}
A federal government is a government that consists of multiple political sub-sovereignties joined together into a large political body.  In the Federalist \#39, James Madison defines a federal government as one where ``the powers operate on the political bodies composing the Confederacy, in their political capacities.''  That is, the larger federal government operates only on the political constituent parts (for example, States), as opposed to on the individual members of the constituent parts (i.e. people).  To give an example, consider five countries that border each other.  Suppose the five bordering countries decide to enter into a federation and create a federal government.  Each of the countries still retains power over its citizens and the federal government cannot legislate directly on the citizens of each of the member countries.  But the federal government can legislate on the countries themselves, such as requiring that the countries impose no import or export taxes when commerce goes to or from another member of the federation.  This is the difference between having the power to operate on the constituent parts of the federation (each of the bordering countries) versus the individual members of the constituent parts (the citizens of the federated countries).

\item \textbf{National}
Conversely, a national government is one where the government can exercise power directly over the \textit{individual} citizens.  In the Federalist \#39, Madison defines a national government as one where ``the powers operate on the individual citizens composing the nation, in their individual capacities..... The idea of a national government involves in it, not only an authority over the individual citizens, but an indefinite supremacy over all persons and things, so far as they are objects of lawful government.''  In a national government, there are no political subdivisions as intermediaries between the national government and its individual citizens or members; instead, the national government can exercise its power directly on the individual citizens over which it has jurisdiction.

\end{itemize}

It is important, though, when discussing the various forms of government to keep in mind that the different forms of government aren't necessarily mutually exclusive, and often times in practice they are mixed.  For example, you can have a democractic republic, or a monarchy mixed with a republic, or a mixed federal and national government that contains elements of a democracy and elements of a republic.

\subsection{Sovereign}

Any discussion on government must include a section on the concept of the ``sovereign.''  
The sovereign is that entity within a society from which all political power or authority emanates.  The sovereign is the fundamental basis and ultimate source and authority for any government's power.

As an example, consider a monarchy.  In a strict monarchy the king\footnote{or queen, as it may be.} is the sovereign.  The king possess or exercises supreme political power and authority and all of the power of the government flows from the monarch.  In order for any action by the government to have legitimacy it must stem, either directly or indirectly, from the monarch.  

However, in a democracy or republic, it is less straightforward who or what the sovereign is.  This is because when a government derives its legitimacy and power from the people, the transfer of power from the people to the government can have different implications.  For example, consider the difference between the concepts of (1) alienation and (2) delegation.

\begin{itemize}
\item \textbf{Alienation} Alienation is act of \textit{fully} transferring ownership of something, such as a right, to another entity.  If, in creating a government, the people have alienated (that is, \textit{fully} transfered) their sovereignty to the government, then after the transfer the government, not the people, is sovereign, and although the people participate in its operation and control its functions, sovereignty is vested in the government, not the people.

\item \textbf{Delegation} Delegation is the act of designating an entity to exercise a particular right or power on behalf of another entity.  In contrast to alienation, if, in creating a government, the people have merely \textit{delegated} instead of \textit{alienated} their sovereign power to the government, then the people remain sovereign and the government is merely an agent or proxy exercising the people's sovereign power on their behalf.
\end{itemize}


\subsection{The Government of the United States}

What form of government does the United States have?  To understand its current form of government, a look back at the history and evolution of the United States government will be instructive.

\subsubsection{History}

Prior to the Declaration of Independence, the original 13 American States existed as colonies of the British Empire, subject to the English sovereign, King George III.  After declaring their independence, the 13 States considered themselves discrete, independent states, separate and distinct from each other and sovereign within the bounds of their own borders.\footnote{See, for example, the Articles of Confederation, Article II (``Each state retains its sovereignty, freedom and independence, and every Power, Jurisdiction and right, which is not by this confederation expressly delegated to the United States, in Congress assembled.'').}  However, despite their valued separateness, the States discovered that the realities of fighting a revolutionary war sometimes required that they act as a single, cohesive unit.  Therefore, in order to fight the revolutionary war and deal with other issues that affected all of the states collectively, the states joined together to form a loose confederation.\footnote{Articles of Confederation, Article III (``The said states hereby severally enter into a firm league of friendship with each other, for their common defence, the security of their Liberties, and their mutual and general welfare, binding themselves to assist each other, against all force offered to, or attacks made upon them, or any of them, on account of religion, sovereignty, trade, or any other pretence whatever.'').}  They embodied the terms of this union in a document entitled ``The Articles of Confederation'' and they called their confederation ``The United States of America.''\footnote{Articles of Confederation, Article I (``The Stile of this confederacy shall be, "The United States of America."'').}  The Articles of Confederation document was the first ``Constitution'' for the United States and the first government formed by the States to address what they considered to be national issues.  However, the government formed by the Articles of Confederation was very weak and had very limited powers.  In fact, the Articles were more of a treaty (or ``firm league of friendship'' as the Articles styled it) between independent nations rather than a formation of a new unified government.\footnote{Articles of Confederation, Article III (``The said states hereby severally enter into a firm league of friendship with each other ... binding themselves to assist each other, against all force offered to, or attacks made upon them, or any of them'').}  The Articles of Confederation had 13 Articles and the text of the second and third Articles read: 

\begin{quote}
Each state retains its sovereignty, freedom, and independence, and every power, jurisdiction, and right, which is not by this Confederation expressly delegated to the United States

The said States hereby severally enter into a firm league of friendship with each other, for their common defense, the security of their liberties, and their mutual and general welfare, binding themselves to assist each other, against all force offered to, or attacks made upon them, or any of them, on account of religion, sovereignty, trade, or any other pretense whatever.
\end{quote}

Practically speaking, the government that the Articles created was a very limited government; it had no executive branch, nor really any judiciary to speak of.  It had a legislative branch, composed of 2 -- 7 delegates from each state.  However, each State had only 1 vote in Congress, regardless of the number of delegates it sent to the Congress.\footnote{Articles of Confederation, Article V (``No State shall be represented in Congress by less than two, nor by more than seven Members; and no person shall be capable of being delegate for more than three years, in any term of six years; nor shall any person, being a delegate, be capable of holding any office under the united states, for which he, or another for his benefit receives any salary, fees or emolument of any kind ...    In determining questions in the united states, in Congress assembled, each state shall have one vote.'').}  Further, the Congress' legislative powers were very limited.  For example, the United States government under the Articles did not even have any real power to raise funds.  While Article 8 said that money was to be supplied by the States in proportion to each State's land values, the United States government had no power to enforce this provision.  Instead, under the Articles the States were responsible for supplying the appropriate funds on their own.\footnote{Articles of Confederation, Article VIII (``All charges of war, and all other expenses that shall be incurred for the common defence or general welfare, and allowed by the united states in congress assembled, shall be defrayed out of a common treasury, which shall be supplied by the several states, in proportion to the value of all land within each state, granted to or surveyed for any Person, as such land and the buildings  and improvements thereon shall be estimated, according to such mode as the united states, in congress assembled, shall, from time to time, direct and appoint. The taxes for paying that proportion shall be laid and levied by the authority and direction of the legislatures of the several states within the time agreed upon by the united states in congress assembled.'').}  If a State did not make the appropriate contribution, the United States government had no real recourse to enforce a State's compliance.

Further, defects such as these were difficult to fix because the Articles could not be amended without unanimous approval by \underline{all} of the States.  And while this functioned (more or less) for the duration of the revolutionary war, afterwards the inherent defects in the Articles, and the difficulty in amending the Articles, began to prove to be a serious impediment to effectively addressing the country's pressing national problems.

\subsubsection{Addressing the Defects of the Articles of Confederation}
As mentioned above, one of the biggest problems with the Articles was that the government had no way to raise money to pay its debts.  The government couldn't impose or enforce a tax.  It could merely outline the voluntary contributions due from each State.  After the war, with revolutionary war loan payments looming and no money to pay the large number of veteran pensions and back payments already outstanding, the inability to lay and enforce taxes left the fledgling government on a very unstable financial foundation.  

To address the defects, a number of men seized the initiative and met at a constitutional convention in Philadelphia in 1787, where they spent the summer working and crafting a replacement for the Articles of Confederation.  The members of the convention struggled with what type of government to make and what to do with the parts that made the current whole.  Should they keep a confederation of independent, sovereign states?  Should they make a national government?  Should they abolish the concept of states entirely?

At the time, the people of the United States did not consider themselves citizens of a national country.  Rather, they considered themselves citizens of their respective States.  They weren't so much ``Americans'' as they were Virginians or New Yorkers or South Carolinians.  Because of this, people were loathe to abolish the concept of the state and to unify into a single country.  But the confederate government under the Articles of Confederation simply did not have the power to be effective, either domestically or in foreign affairs, because it lacked the ability to enforce its will on the States individually or the people of the States directly.  

\subsubsection{Splitting the Atom}

In the end, the constitutional convention in Philadelphia settled on a compromise.  Instead of a pure confederacy or a pure national government, they created an amalgamation of a federal government and a national government.  Under the Constitution, the United States government was meant have to elements of both types of governments.  In some ways, the United States government would have the qualities of a national government -- that is, it could operate and legislate over not just the states themselves,but also over the \textit{citizens} of the states directly.\footnote{Such as by imposing taxes.}  But in some ways, the United States government would retain the elements and limitations of a federal government because the States would continue to not only exist, but to remain sovereign in areas that the United States government did not have authority over.  In the Federalist \#39 James Madison talks about the elements of federalism and nationalism that are embodied in the United States government.

\begin{quote}
The proposed Constitution, therefore, … is, in strictness, neither a national nor a federal Constitution, but a composition of both. 
In its foundation it is federal, not national; in the sources from which the ordinary powers of the government are drawn, it is partly federal and partly national; in the operation of these powers, it is national, not federal; in the extent of them, again, it is federal, not national; and, finally, in the authoritative mode of introducing amendments, it is neither wholly federal nor wholly national.
\end{quote}

In making a national government, but still keeping the states, the United States created something never before seen in the governments of the world.
The people of sovereign states joined together to create a new government which exercised the power of sovereignty in some areas (i.e. a national government), while the states retained and exercised sovereignty in other areas (i.e. a federal government).  Justice Anthony Kennedy refers to this as ``splitting the atom of sovereignty.''  In his concurrence in \textit{U.S. Term Limits, Inc. v. Thornton}, 514 U.S. 779 (1995), Justice Kennedy said: 

\begin{quote}
Federalism was our Nation's own discovery.  The Framers split the atom of sovereignty.  It was the genius of their idea that our citizens would have two political capacities, one state and one federal, each protected from incursion by the other.  The resulting Constitution created a legal system unprecedented in form and design, establishing two orders of government, each with its own direct relationship, its own privity, its own set of mutual rights and obligations to the people who sustain it and are governed by it.
\end {quote}

So how exactly did the Constitution split the power between the federal government and the States?  Succinctly, it did it by starting with fundamentally different presumptions as to each.  Consider the following 2 premises of government:

\begin{enumerate}
\item Start with the general presumption that the government has all power, and is only limited by specific and explicit restrictions.
\item Start with the general presumption that the government does not have any power, and is only able to exercise those powers specifically and explicitly designated.
\end{enumerate}

Under the Constitution, the former presumption is assigned to the States and the latter to the federal government.
That is, the State governments are generally presumed to have power over all things unless explicitly circumscribed by the United States Constitution or their own state constitutions, whereas the United States government is generally presumed to \textit{not} have power over a particular subject matter unless such power has been explicitly delegated to it by the United States Constitution.  This concept will be elaborated on further in later chapters.

\section{Control of Government}

Despite the practical necessity of and benefits provided by government, the United States puts explicit restraints on the power of government.  But, if government is a necessary thing that provides a good to society, why do we restrict it?  Succinctly, to prevent tyranny and protect the freedom of individuals.  James Madison discusses in the Federalist \#51.

\begin{quote}
If men were angels, no government would be necessary. If angels were to govern men, neither external nor internal controls on government would be necessary. In framing a government which is to be administered by men over men, the great difficulty lies in this: you must first enable the government to control the governed; and in the next place oblige it to control itself.
\end{quote}

Madison acknowledges that government, which is instituted to control the bad acts of people, is made up of people.  Thus, the people in government need to be restrained just as much as those outside of it.  And the danger of leaving those in government unrestrained is that it leads to a consolidation of power, which Madison says is the very definition of tyranny.

\begin{quote}
The accumulation of all powers, legislative, executive, and judiciary, in the same hands, whether of one, a few, or many, and whether hereditary, self appointed, or elective, may justly be pronounced the very definition of tyranny.\footnote{James Madison, Federalist \#47}
\end{quote}

To address the danger of consolidation of power, Madison proposes a solution -- split up the powers of government.  Diffuse the government's powers into different departments, set them at odds with each other, and give them the means to resist attempts at encroachments by other departments.  Or, as Madison says: ``Ambition must be made to counteract ambition.''

\begin{quote}
[T]he great security against a gradual concentration [of] … power ... in the same department consists in giving to those [departments] the … means and personal motives to resist encroachments [by] the others ... Ambition must be made to counteract ambition. The interest of the man must be connected with the constitutional rights of the place. It may be a reflection on human nature, that such devices should be necessary to control the abuses of government. But what is government itself, but the greatest of all reflections on human nature?\footnote{James Madison, Federalist \#51}
\end{quote}

Thus, we split apart the powers of government into different hands.  And in the United States we do this in multiple ways.  First, we separate the powers of government by allocating some power to the federal government and some power to the states.  Further, in both the federal and state governments, we divide the lawmaking power of the government (the legislative branch) from the enforcement power of the government (the executive branch) and divide both of those from the judging power of the government (the judicial branch).  Moreover, among the states, we then limit each of the States' powers by bounding them by geographic boundaries.

Thus we have the ambitions of the men in the federal government pitted against those in the state governments, and the ambitions of those men in the state governments pitted against those of the federal government as well as those of the other state governments.  And finally, we have the ambitions of the men in the federal government and each of the individual state governments pitted against themselves internally by dividing each individual government into three distinct branches -- legislative, executive and judicial.

And for over 200 years the government of the United States has functioned in this way.