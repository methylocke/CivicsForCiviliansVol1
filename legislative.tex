% ******************************************************** %
% Copyright © Jonathan Gitlin 2011 – 2013
%
% This work is licensed under the Creative Commons 
% Attribution-NonCommercial-ShareAlike 3.0 Unported License
%
% For More Information See the Creative Commons Website
% http://creativecommons.org
%
% For A human readable summary, See:
% http://creativecommons.org/licenses/by-nc-sa/3.0/
%
% For the Full Text of the License:
% http://creativecommons.org/licenses/by-nc-sa/3.0/legalcode
%
% No Claim Is Made To Original Works 
% Of Other Sources Cited Within This Work
%
% ******************************************************** %

\chapter{Legislative Branch}

\section {Purpose of Chapter}
This chapter covers the legislative branch of the United States, that branch of the United States government that makes its laws.

\section{Creation}
The Legislative Branch is created in Article I of the Constitution.  Per Article I, Section I:

\begin{quote}
All legislative powers herein granted shall be vested in a Congress of the United States, which shall consist of 
a Senate and House of Representatives.
\end{quote}

Section I of Article I of the Constitution does two things.  First, it establishes that the legislative power of the United States is vested in the United States Congress.  Second, it divides Congress into two houses: (1) a Senate; and (2) a House of Representatives.

\section{Legislative Power}

\subsubsection{What is Legislative Power}
Legislative power is the power to make laws.  Laws are the means by which the United States government enforces its will on the governed.  Accordingly, as Alexander Hamilton describes in the Federalist \#15, all laws must carry with them associated sanctions for disobedience.  Otherwise, such ``laws'' would amount to no more than mere recommendations or suggestions.

\begin{quote}
Why has government been instituted at all? Because the passions of men will not conform to the dictates of reason and justice, without constraint ... 
Government implies the power of making laws. It is essential to the idea of a law, that it be attended with a sanction; or, in other words, a penalty or punishment for disobedience. If there be no penalty annexed to disobedience, the resolutions or commands which pretend to be laws will, in fact, amount to nothing more than advice or recommendation.\footnote{Alexander Hamilton, The Federalist \#15}
\end{quote}

The power to make the laws of the United States is the power to define the will of the United States government, and accordingly the scope and boundaries of United States society.

\section{Bicameral Legislature}
After delegating the Legislative power of the United States to Congress, the Constitution goes on to create a \textit{bicameral} legislature, dividing the Legislative branch into two houses -- a House of Representatives and a Senate.  Splitting Congress into two separate houses accomplishes several different objectives.

First, each of the Houses was meant to serve a different purpose.  The House of Representatives was meant to be the people's house, with members elected directly by the populace and allocation of representatives to correspond to the populations of the states.  Conversely, the Senate was meant to have a closer relationship to the States as political entities, with members elected by the State legislatures and equal representation for all states.  By splitting Legislative Branch into two different houses the Constitution could accomplish both of these purposes.

Second, splitting the Legislative branch into two separate Houses serves to further effectuate a separation of powers.  In the Federalist \#48, James Madison warns that in the United States, with its carefully limited executive department, and a legislative branch that is numerous enough to feel the passions of a mob, but not so numerous as to be incapable of accomplishing anything, the legislative branch is the department against which the most precautions ought to be exercised.

\begin{quote}
But in a representative republic, where the executive magistracy is carefully limited; both in the extent and the duration of its power; and where the legislative power is exercised by an assembly, which is inspired, by a supposed influence over the people, with an intrepid confidence in its own strength; which is sufficiently numerous to feel all the passions which actuate a multitude, yet not so numerous as to be incapable of pursuing the objects of its passions, by means which reason prescribes; it is against the enterprising ambition of this department that the people ought to indulge all their jealousy and exhaust all their precautions.\footnote{James Madison, The Federalist \#48}
\end{quote}

Thus, as separating the United States government's powers into separate branches is intended to protect against encroachments by the government on the liberty of the citizenry, the Constitution takes this one step further by splitting the Legislative Branch into two separate Houses.  As covered in the previous chapter, this separation of the legislative power serves to guard against a consolidation of power, especially by what Madison through to be the most dangerous branch.  Practically speaking, the separation puts the two Houses somewhat at odds (as competing parts of a greater whole often find themselves) but still forces them to find a way to cooperate with each other (as Congress cannot pass a law without it passing both houses in identical form, and thus Congress cannot exercise its legislative power without cooperation between the two).

\section{House of Representatives}

\subsubsection{Creation}
The House of Representatives is the first of the two Houses of Congress created by the Constitution and is created in Article I, Section 2 of the Constitution.

\subsubsection{Why We Have It}
The House of Representatives was meant to embody the principles of representative government, and to achieve two major goals:

\begin{enumerate}
\item \textbf{People's House} The House of Representatives was meant to be a house of the People, involving direct representation.  Madison described this in the Federalist \#52: ``As it is essential to liberty that the government in general should have a common interest with the people, so it is particularly essential that the branch of it under consideration should have an immediate dependence on, and an intimate sympathy with, the people.''  The intent was for the House to be directly connected to and have a strong dependence on the People.
\item \textbf{Favor the More Populous States} The House was meant to favor the larger, more populous states of the union, as representation would be allocated according to population.  Accordingly the larger, more populous states would have greater representation.  Madison described this in the Federalist \#58: ``There is a peculiarity in the federal Constitution which insures a watchful attention in a majority both of the people and of their representatives to a constitutional augmentation of the latter.  The peculiarity lies in this, that one branch of the legislature is a representation of citizens, the other of the States: in the former, consequently, the larger States will have most weight; in the latter, the advantage will be in favor of the smaller States.  From this circumstance it may with certainty be inferred that the larger States will be strenuous advocates for increasing the number and weight of that part of the legislature in which their influence predominates.''
\end{enumerate}

\subsubsection{Constitutional Requirement of Proportional Representation}

Article I, Section 2 of the Constitution establishes that representation in the House of Representatives shall be divided up proportionally amongst the States based on their respective populations:

\begin{quote}
Representatives and direct taxes shall be apportioned among the several states which may be included within this union, according to their respective numbers, which shall be determined by adding to the whole number of free persons, including those bound to service for a term of years, and excluding Indians not taxed, three fifths of all other Persons.
\end{quote}

And Article I, Section 2 establishes the formula for determining each State's population for apportioning representation in the House of Representatives: each State's population is determined by counting all of people in the State, except (1) Native Americans are not to be included in the count and (2) Slaves are only to be counted as 3/5 of a person.\footnote{Article I, Section 2 (``which shall be determined by adding to the whole number of free persons, including those bound to service for a term of years, and excluding Indians not taxed, three fifths of all other Persons'').}

This formula has two oddities in that it (1) excludes Native Americans and (2) counts slaves as 3/5 of a person.  Native Americans were considered citizens of foreign nations, so they were not included in the population of a State.  Slaves were only counted as 3/5 a person as a compromise between the Northern states and the Southern states.  The Southern states wanted slaves to be counted in their States' populations for allocation of representatives in the House.  However the Northern States had serious problems with this for a multitude of reasons.  Among other objections, the North accused the South of perpetrating a double standard.  Despite the fact that the Southern states considered slaves to be ``property'' and not people (just like horses and other livestock), the South still wanted the slaves counted as ``people'' for representation in the House.\footnote{It goes without saying that slaves could not vote.}  A compromise was reached, and the Slaves were counted as 3/5 of a person for representative purposes.


\subsubsection{How the Size of the House and Allocation of Representation are Set}

The Constitution specifies the original number of representatives in the House and their allocation among of the States.\footnote{Article I, Section 2 (``until such enumeration shall be made, the state of New Hampshire shall be entitled to chuse three, Massachusetts eight, Rhode Island and Providence Plantations one, Connecticut five, New York six, New Jersey four, Pennsylvania eight, Delaware one, Maryland six, Virginia ten, North Carolina five, South Carolina five, and Georgia three.''}

This original allocation was to stay in effect until the first census was taken.\footnote{Article I, Section 2 (``and until such enumeration shall be made, the State of New Hampshire shall be entitled to ...'').}  However, once the Census was taken, the Constitution but did not specify (1) how many representatives total should be in the House or (2) how they should be allocated among the States, other than limiting representation to no more than 1 representative for every 30,000 people, requiring that each state have at least 1 representative, and requiring that representation by proportional to population.\footnote{Article I, Section 2 (``The number of Representatives shall not exceed one for every thirty thousand, but each state shall have at least one Representative'')}  Therefore, the total number of representatives in the House and the proportional distribution of representation among the States is set by Federal law.

\subsubsection{Size of House}

The size of the House (i.e. the total number of Representatives in the House) is set by Public Law 62-5 and the Permanent Apportionment Act of 1929.\footnote{History, Art \& Archives, U.S. House of Representatives, ``The Permanent Apportionment Act of 1929,'' http://history.house.gov/HistoricalHighlight/Detail (February 03, 2013)}  Per these laws, there are 435 representatives allocated amongst the states.  In addition, there are five non-voting delegates that represent the District of Columbia, the U.S. Virgin Islands, Guam, American Samoa, and the Commonwealth of the Northern Marian Islands.  Further, Puerto Rico also elects a non-voting Resident Commissioner to serve a four-year term.

\subsubsection{Allocation of Representation}
Per Federal law, the number of representatives allocated among the states is set at 435, and those 435 are allocated among the states by what is called the ``Method of Equal Proportion.''  This process begins with the 10-year census that is mandated by the Constitution in Article I, Section 2.\footnote{Article I, Section 2 (``The actual Enumeration shall be made within three years after the first meeting of the Congress of the United States, and within every subsequent term of ten years, in such manner as they shall by law direct.'')}
The procedure and method of taking the census is laid out in Title 13 of the United States Code.  Once the census is complete, 2 U.S.C. \S\ 2a requires that the President transmit to Congress the results of the census, and the number of representatives each state is entitled to per the Method of Equal Proportions.\footnote{2 USC \S\ 2a(a) (``On the first day, or within one week thereafter, of the first regular session of the Eighty-second Congress and of each fifth Congress thereafter, the President shall transmit to the Congress a statement showing the whole number of persons in each State, excluding Indians not taxed, as ascertained under the seventeenth and each subsequent decennial census of the population, and the number of Representatives to which each State would be entitled under an apportionment of the then existing number of Representatives by the method known as the method of equal proportions, no State to receive less than one Member.''); 

2 USC \S\ 2b (``Each State shall be entitled, in the Seventy-eighth and in each Congress thereafter until the taking effect of a reapportionment under a subsequent statute or section 2a of this title, to the number of Representatives shown in the statement transmitted to the Congress on January 8, 1941, based upon the method known as the method of equal proportions, no State to receive less than one Member.'')}
Each state is entitled receive at least one representative in the House.\footnote{Article I, Section 2 (``each state shall have at least one Representative''); 2 USC \S\ 2a(b) (``no State to receive less than one Member'').}

Once Congress receives the census and allocation, within 15 days, the Clerk of the House of Representatives must transmit to the executive of each state (i.e. the governor) a certificate of the number of representatives that that state is entitled to under the new census.\footnote{2 USC \S\ 2a(b) (``Each State shall be entitled, in the Eighty-third Congress and in each Congress thereafter until the taking effect of a reapportionment under this section or subsequent statute, to the number of Representatives shown in the statement required by subsection (a) of this section, no State to receive less than one Member. It shall be the duty of the Clerk of the House of Representatives, within fifteen calendar days after the receipt of such statement, to send to the executive of each State a certificate of the number of Representatives to which such State is entitled under this section. In case of a vacancy in the office of Clerk, or of his absence or inability to discharge this duty, then such duty shall devolve upon the Sergeant at Arms of the House of Representatives.'')}
Once this is done, each State must pass a law dividing itself into districts, the number of districts corresponding to the number of representatives allocated to the State under the census.\footnote{2 USC \S\ 2c (``In each State entitled in the Ninety-first Congress or in any subsequent Congress thereafter to more than one Representative under an apportionment made pursuant to the provisions of section 2a (a) of this title, there shall be established by law a number of districts equal to the number of Representatives to which such State is so entitled, and Representatives shall be elected only from districts so established, no district to elect more than one Representative (except that a State which is entitled to more than one Representative and which has in all previous elections elected its Representatives at Large may elect its Representatives at Large to the Ninety-first Congress).''); 2 USC \S\ 2a(c) (``Until a State is redistricted in the manner provided by the law thereof after any apportionment, the Representatives to which such State is entitled under such apportionment shall be elected in the following manner: (1) If there is no change in the number of Representatives, they shall be elected from the districts then prescribed by the law of such State, and if any of them are elected from the State at large they shall continue to be so elected; (2) if there is an increase in the number of Representatives, such additional Representative or Representatives shall be elected from the State at large and the other Representatives from the districts then prescribed by the law of such State; (3) if there is a decrease in the number of Representatives but the number of districts in such State is equal to such decreased number of Representatives, they shall be elected from the districts then prescribed by the law of such State; (4) if there is a decrease in the number of Representatives but the number of districts in such State is less than such number of Representatives, the number of Representatives by which such number of districts is exceeded shall be elected from the State at large and the other Representatives from the districts then prescribed by the law of such State; or (5) if there is a decrease in the number of Representatives and the number of districts in such State exceeds such decreased number of Representatives, they shall be elected from the State at large.'')}
Each representative is elected to the House of Representatives via its district, with only one representative per district allowed.\footnote{2 USC \S\ 2c (``In each State entitled in the Ninety-first Congress or in any subsequent Congress thereafter to more than one Representative under an apportionment made pursuant to the provisions of section 2a (a) of this title, there shall be established by law a number of districts equal to the number of Representatives to which such State is so entitled, and Representatives shall be elected only from districts so established, no district to elect more than one Representative (except that a State which is entitled to more than one Representative and which has in all previous elections elected its Representatives at Large may elect its Representatives at Large to the Ninety-first Congress).'').}
Under certain circumstances, States may, however, choose to have one or more of their representatives elected ``at large.''\footnote{2 USC \S\ 2a(c) (``Until a State is redistricted in the manner provided by the law thereof after any apportionment, the Representatives to which such State is entitled under such apportionment shall be elected in the following manner: (1) If there is no change in the number of Representatives, they shall be elected from the districts then prescribed by the law of such State, and if any of them are elected from the State at large they shall continue to be so elected; (2) if there is an increase in the number of Representatives, such additional Representative or Representatives shall be elected from the State at large and the other Representatives from the districts then prescribed by the law of such State; (3) if there is a decrease in the number of Representatives but the number of districts in such State is equal to such decreased number of Representatives, they shall be elected from the districts then prescribed by the law of such State; (4) if there is a decrease in the number of Representatives but the number of districts in such State is less than such number of Representatives, the number of Representatives by which such number of districts is exceeded shall be elected from the State at large and the other Representatives from the districts then prescribed by the law of such State; or (5) if there is a decrease in the number of Representatives and the number of districts in such State exceeds such decreased number of Representatives, they shall be elected from the State at large.'')}

\subsubsection{Qualifications of Members}
The Constitution sets out three requirements for a person to be a member of the House of Representatives:
\begin{itemize}
\item First, the person must be at least twenty five (25) years old.\footnote{Article I, Section 2 (``No person shall be a Representative who shall not have attained to the age of twenty five years'')}
\item Second, the person must have been a citizen of the United States for seven years.\footnote{Article I, Section 2 (``No person shall be a Representative who shall not have ... been seven years a citizen of the United States'')}
\item Third, the person must be an inhabitant of the State he is elected from.\footnote{Article I, Section 2 (``No person shall be a Representative ... who shall not, when elected, be an inhabitant of that state in which he shall be chosen.'')}
\end{itemize}


\subsubsection{Election and Tenure of Members}

As mandated by Article I, Section 2 of the Constitution and enacted by Federal law in Title 2 of the United States Code, members of the House of Representatives are elected every two years and serve two year terms.\footnote{Article I, Section 2 (``The House of Representatives shall be composed of members chosen every second year by the people of the several states''); 2 USC \S\ 7 (``The Tuesday next after the 1st Monday in November, in every even numbered year, is established as the day for the election, in each of the States and Territories of the United States, of Representatives and Delegates to the Congress commencing on the 3d day of January next thereafter.'').}  In the Federalist \#52, James Madison describes why the Constitution sets the tenure of each term of the members of the House of Representatives at two years.

\begin{quote}
As it is essential to liberty that the government in general should have a common interest with the people, so it is particularly essential that the branch of it under consideration should have an immediate dependence on, and an intimate sympathy with, the people. Frequent elections are unquestionably the only policy by which this dependence and sympathy can be effectually secured. 
\end{quote}

Madison says that the two year terms, and accordingly the associated ``frequent'' elections, are required to keep the members of the House dependent on the citizenry so that the House of Representatives can retain its character as the ``People's Branch'' of Congress.

As to who can vote for a member of the House of Representatives, the Constitution requires that the qualifications in a State for voting for a member of the House of Representatives must be the same as the qualifications in that State for voting for the most populous branch of that State's legislature.\footnote{Article I, Section 2 (``the electors in each state shall have the qualifications requisite for electors of the most numerous branch of the state legislature.'')}

As to the particulars of the elections, per Article I, Section 4 of the Constitution, State law generally determines the time, place, and manner of electing Representatives, but Congress can modify these regulations or make their own.\footnote{Article I, Section 4 (``The times, places and manner of holding elections for Senators and Representatives, shall be prescribed in each state by the legislature thereof; but the Congress may at any time by law make or alter such regulations, except as to the places of choosing Senators.'')}
For example, per 2 USC \S\ 7 elections for the House of Representatives are held every two years in even-numbered years (such as 1982, 1984, 1986, etc.) on the Tuesday following the first Monday in November.\footnote{2 USC \S\ 7 (``The Tuesday next after the 1st Monday in November, in every even numbered year, is established as the day for the election, in each of the States and Territories of the United States, of Representatives and Delegates to the Congress commencing on the 3d day of January next thereafter.'').}\footnote{So, for example, if November 1 were a Tuesday, then the election would not take place until November 8, because the first Monday in November would be November 7.}

Additionally, Federal Law requires that all votes for House members must be made in either in written or printed form, or by an approved voting machine.\footnote{2 USC \S\ 9 (``All votes for Representatives in Congress must be by written or printed ballot, or voting machine the use of which has been duly authorized by the State law; and all votes received or recorded contrary to this section shall be of no effect.'').}


\subsubsection{Officers}
The House of Representatives has various officers, such as the Speaker of the House, and the Constitution assigns to the House of Representatives the power to choose its own officers.\footnote{Article I, Section 2 (``The House of Representatives shall choose their speaker and other officers'').}

\section{Senate}

\subsubsection{Creation}
The Senate is the second of the two Houses of Congress created by the Constitution and is created in Article I, Section 3 of the Constitution.


\subsubsection{Why We Have It}
The purpose of the Senate, as opposed to the House of Representatives, was to have an ``upper house'' in Congress, one which was meant to counterbalance the attributes that attended a house of congress (the House of Representatives) whose members were more closely aligned with a populist and democratic ideal.  The Federalist \#62 lays out a number of the ideals that were meant to be embodied in the Senate:

\begin{quote}
the nature of the senatorial trust, which, requiring greater extent of information and stability of character ... In this spirit it may be remarked, that the equal vote allowed to each State is at once a constitutional recognition of the portion of sovereignty remaining in the individual States, and an instrument for preserving that residuary sovereignty ... Another advantage accruing from this ingredient in the constitution of the Senate is, the additional impediment it must prove against improper acts of legislation. No law or resolution can now be passed without the concurrence, first, of a majority of the people, and then, of a majority of the States ... a senate, as a second branch of the legislative assembly, distinct from, and dividing the power with, a first, must be in all cases a salutary check on the government. It doubles the security to the people, by requiring the concurrence of two distinct bodies in schemes of usurpation or perfidy, where the ambition or corruption of one would otherwise be sufficient ... The necessity of a senate is not less indicated by the propensity of all single and numerous assemblies to yield to the impulse of sudden and violent passions, and to be seduced by factious leaders into intemperate and pernicious resolutions … that a body which is to correct this infirmity ought itself to be free from it, and consequently ought to be less numerous. It ought, moreover, to possess great firmness, and consequently ought to hold its authority by a tenure of considerable duration ... Another defect to be supplied by a senate lies in a want of due acquaintance with the objects and principles of legislation. It is not possible that an assembly of men called for the most part from pursuits of a private nature, continued in appointment for a short time, and led by no permanent motive to devote the intervals of public occupation to a study of the laws, the affairs, and the comprehensive interests of their country, should, if left wholly to themselves, escape a variety of important errors in the exercise of their legislative trust ... The mutability in the public councils arising from a rapid succession of new members, however qualified they may be, points out, in the strongest manner, the necessity of some stable institution in the government. Every new election in the States is found to change one half of the representatives. From this change of men must proceed a change of opinions; and from a change of opinions, a change of measures. But a continual change even of good measures is inconsistent with every rule of prudence and every prospect of success.
\end{quote}

The salient points outlined by Madison are:
\begin{enumerate}
\item \textbf{Separation of Powers} Because Congress was the branch of government ``that the people ought to indulge all their jealousy and exhaust all their precautions,''\footnote{Federalist \#48} the legislative power was split between two houses.  The Senate was the second of the two.
\item \textbf{Sovereignty Embodied in States} Just as the House was meant to have an intimate connection to the People, the Senate was, originally, intended to have a stronger connection with the States as discrete political units.
\item \textbf{An Upper House} The intent of the Senate was also to have a more sober and somber legislative body, with the intent that this would put a check on the potential excesses of the populist branch (the House) and the predilection of a numerous assembly to yield to a mob mentality.
\item \textbf{Stability} Further, as the House could potentially change its entire membership every two years, there was a desire to have a house of Congress whose membership was less volatile, to provide a more stable half of Congress.
\item \textbf{Favor the Less Populous States} Just as the House would favor the more populous states with its proportional representation, the Senate would provide a balance for the less populous states.  Since each state had equal representation in the Senate, the smaller states had just as much power in the Senate as the larger states, and the smaller states couldn't be overwhelmed in the voting process by the larger states simply on account of the size of their population.
\end{enumerate}


\subsubsection{Size of Senate}
The size of the Senate is set by the Constitution.  Article I, Section 3 and the 17th Amendment establish that each State shall have 2 senators in the Senate, and that each Senator shall have one vote.\footnote{Article I, Section 3 (``The Senate of the United States shall be composed of two Senators from each state, chosen by the legislature thereof, for six years; and each Senator shall have one vote.''); Seventeenth Amendment (``The Senate of the United States shall be composed of two Senators from each state, elected by the people thereof, for six years; and each Senator shall have one vote.'').}
The United States has 50 states, so with two Senators per state the Senate currently has 100 members.

\subsubsection{Qualifications of Senators}
The Constitution sets out three requirements for a person to be a member of Senate

\begin{itemize}
\item First, the person must be at least thirty (30) years old.\footnote{Article I, Section 3 (``No person shall be a Senator who shall not have attained to the age of thirty years'').}
\item Second, the person must have been a citizen of the United States for nine years.\footnote{Article I, Section 3 (``No person shall be a Senator who shall not have ... been nine years a citizen of the United States'').}
\item Third, the person must be an inhabitant of the state he is elected from.\footnote{Article I, Section 3 (``No person shall be a Senator ... who shall not, when elected, be an inhabitant of that state for which he shall be chosen.'').}
\end{itemize}

\subsubsection{Who Elects Senators}
Originally, under Article I, Section 3 of the Constitution, Senators weren't elected by the people as they are today.  Instead, they were chosen by the States' legislatures.  The people of a state had no direct say in who their senator was.  Instead, they chose the members of their state legislatures, which in turn chose the Senators to represent their States.\footnote{Article I, Section 3 (``The Senate of the United States shall be composed of two Senators from each state, chosen by the legislature thereof, for six years; and each Senator shall have one vote.'').}
However, the 17th Amendment changed this; no longer were Senators chosen by state legislatures.  Instead they are elected directly by the people of the State, like the representatives in the House.\footnote{Seventeenth Amendment (``The Senate of the United States shall be composed of two Senators from each state, elected by the people thereof, for six years; and each Senator shall have one vote.'').}
Under the 17th Amendment the qualifications for voting in a State for a Senator are, similar to those for voting for a Representative in the House, the same as the qualifications for voting for the most numerous branch of the state legislatures.\footnote{Seventeenth Amendment (``The electors in each state shall have the qualifications requisite for electors of the most numerous branch of the state legislatures.'').}

\subsubsection{Length of Term}
Per Article I, Section 3 and the 17th Amendment, Senators serve for a term of 6 years.\footnote{Article I, Section 3 (``The Senate of the United States shall be composed of two Senators from each state, chosen by the legislature thereof, for six years; and each Senator shall have one vote.''); 
Seventeenth Amendment (``The Senate of the United States shall be composed of two Senators from each state, elected by the people thereof, for six years; and each Senator shall have one vote.'').}

\subsubsection{Classes}
Unlike the House of Representatives, the Senate was intended to have an enduring character between elections.  In this vein, the Constitution divides the Senators into 3 different ``classes.'' The elections for the classes are staggered, so only one class of Senators comes up for election every two years.  Thus, unlike the House of Representatives which could potentially change its entire membership in a single election, at most only one third of the Senate could be replaced every two years.\footnote{Article I, Section 3 (``Immediately after they shall be assembled in consequence of the first election, they shall be divided as equally as may be into three classes. The seats of the Senators of the first class shall be vacated at the expiration of the second year, of the second class at the expiration of the fourth year, and the third class at the expiration of the sixth year, so that one third may be chosen every second year;'').}

\subsubsection{Officers}
Like the House of Representatives, the Senate has its own officers and, except for the President of the Senate, the Constitution assigns to the Senate the power to choose its own officers.\footnote{Article I, Section 3 (``The Senate shall choose their other officers, and also a President pro tempore, in the absence of the Vice President, or when he shall exercise the office of President of the United States.'').}
The Constitution assigns the office of the President of the Senate, however, to the Vice President of the United States..\footnote{Article I, Section 3 (``The Vice President of the United States shall be President of the Senate'').}
The Vice President, however, can't vote in the Senate, unless there is a tie in a vote, at which point he can cast the tie-breaker vote.\footnote{Article I, Section 3 (``The Vice President of the United States ... shall have no vote, unless they be equally divided.'').}

\section{Removal of Members of Congress}
Per the Constitution, once someone becomes a member of Congress, to remove either a Representative or Senator from office requires a two-thirds vote of that person's respective House.\footnote{Article I, Section 5 (``Each House may determine the rules of its proceedings, punish its members for disorderly behavior, and, with the concurrence of two thirds, expel a member.'').}
Further, once a person has been duly elected to Congress, Congress cannot exclude that member \textit{prior} to that member taking his or her seat.  Congress can only expel one of its members \textit{after} that member has been seated.\footnote{\textit{Powell v. McCormack}, 395 U.S. 486, 550 (1969)(``Therefore, we hold that, since Adam Clayton Powell, Jr., was duly elected by the voters of the 18th Congressional District of New York and was not ineligible to serve under any provision of the Constitution, the House was without power to exclude him from its membership.'').}

\section{Power of Congress}

\subsection{Government of Limited Powers}
The United States government is a government of limited powers.  Unlike the State governments which operate on a presumption of power, whereby the power to legislative is presumed unless restricted, the United States government operates on an opposite presumption, whereby the power to legislate is presumed lacking unless delegated by the Constitution or its amendments.\footnote{\textit{United States v. Lopez}, 514 US 549, 552 (1995) (``We start with first principles. The Constitution creates a Federal Government of enumerated powers. See Art. I, \S\ 8. As James Madison wrote: `The powers delegated by the proposed Constitution to the federal government are few and defined. Those which are to remain in the State governments are numerous and indefinite.' The Federalist No. 45, pp. 292-293 (C. Rossiter ed. 1961).''); \textit{United States v. Butler}, 297 US 1, 66 (1936)(``As Story says: `The Constitution was, from its very origin, contemplated to be the frame of a national government, of special and enumerated powers, and not of general and unlimited powers.''').}
Thus, to determine what powers the United States government possesses, one must always begin by referencing to the Constitution.

\subsection{Power to Make Law}
The very first grant of power in the Constitution is the power to make laws.  Article I, Section I says: ``All legislative powers herein granted shall be vested in a Congress of the United States.''  What subject matters this grant of legislative power actually entails, though, is spread throughout the rest of the Constitution.

\subsection{Enumerated Powers In Original Body of Constitution}
The Constitution grants Congress the power to legislate on various subject matters throughout the body of the Constitution itself.  However, the main grant of legislative power in the original body of the Constitution is in Article I, Section 8.  Some of the legislative powers granted in the original body of the Constitution are covered below.

\subsubsection{Tax and Spend}
The first power granted in Article I, Section 8 is the power to tax and spend: ``The Congress shall have power to lay and collect taxes, duties, imposts and excises, to pay the debts and provide for the common defense and general welfare of the United States.''

This grant contains a number of powers, including the power to tax and the power to spend money to (1) pay the debts of the United States, (2) provide for the common defense of the United States, and (3) provide for the general welfare of the United States.

This last power is usually referred to as the ``General Welfare Clause'' and has been generally interpreted by the Supreme Court to give Congress very broad latitude in determining how to spend the money of the United States Government.

\subsubsection{Borrow Money and Incur Debt}
The next power the Constitution grants in Article I, Section 8 is the power to borrow money and incur debt: ``To borrow money on the credit of the United States;''

\subsubsection{Regulation of International and Interstate Commerce}
The next power the Constitution grants in Article I, Section 8 is the power to regulate commerce (1) with foreign nations and (2) among ``the several states.''\footnote{Article I, Section 8 (``To regulate commerce with foreign nations, and among the several states, and with the Indian tribes;'').}
The phrase ``the several States'' refers to the States in the Union, and this grant of power is generally referred to as the ``Interstate Commerce Clause'' and is what allows Congress to legislate about matters that affect interstate commerce.

\subsubsection{Powers, Generally}
Article I, Section 8 also grants Congress power to legislate numerous other subject matters, including immigration and naturalization, bankruptcies, money, weights and measures, a postal system, intellectual property (including patents, copyrights, etc), the creation of the judicial system of the United States, maritime laws, declarations of war, creating and maintaining an army and navy, and to ``make all laws which shall be necessary and proper for carrying into execution the foregoing powers, and all other powers vested by this Constitution in the government of the United States, or in any department or officer thereof.''
This last power (and the language granting Congress this power) is referred to as the ``Necessary and Proper'' clause and, like the Interstate Commerce Clause and the General Welfare Clause, has been interpreted to allow Congress fairly broad latitude in making laws to carry out and exercise the other previously-enumerated powers.

\subsubsection{Presidential Succession}
Article II, Section 1 gives Congress the power to set the order of presidential succession after the vice-president.

\subsection{Enumerated Powers In Amendments}
Some of the amendments subsequent to the adoption of the Constitution gave Congress powers beyond the original enumerated powers.
A number of the amendments enact fundamental structural and legal shifts in the United States and contain language granting Congress the power to enforce the provisions of that particular amendment.

For example, the Thirteenth Amendment, which abolished slavery in the United States, gives Congress the power to enforce the amendment by law:

\begin{quote}
\textit{Section 1.} Neither slavery nor involuntary servitude, except as a punishment for crime whereof the party shall have been duly convicted, shall exist within the United States, or any place subject to their jurisdiction.  

\textit{Section 2.} Congress shall have power to enforce this article by appropriate legislation.
\end{quote}

The Fourteenth Amendment, which, among other things, made citizenship a birthright in the United States and imposed upon the states the requirements of Due Process and Equal Protection, has a similar provision in Section 5 of the amendment that allows the Congress to enforce the amendment's provisions by law.\footnote{Fourteenth Amendment, Section 5 (``The Congress shall have power to enforce, by appropriate legislation, the provisions of this article.'').}

Not all of the amendments that expand Congress' powers are structured this way, however.  Some are direct grants of legislative power instead.  For example, the Sixteenth Amendment, which grants Congress the power to tax income, is such a direct grant of legislative power:

\begin{quote}
The Congress shall have power to lay and collect taxes on incomes, from whatever source derived, without apportionment among the several states, and without regard to any census or enumeration.
\end{quote}

\section{Impeachment}
Congress has the power remove members of the United States government from office that would otherwise be protected from removal.
Under the Constitution, this is a 2-step process, and the Constitution divides up the two steps between the two Houses of Congress.

\subsubsection{Impeachment}

First, Congress must impeach, that is accuse or bring charges against, the person.  The Constitution assigns the power to impeach to the House of Representatives.

Article I, Section 2 of the Constitution delegates the power to impeach, that is, bring charges, to the House of Representatives.\footnote{Article I, Section 2 (``The House of Representatives … shall have the sole power of impeachment.'').}  If the House of Representatives votes to impeach, then the process moves on to the Senate.

\subsubsection{Conviction}
Second, Congress must try the impeachment, that is, hold a trial on the accusation by the House of Representatives, and either convict or acquit the person.  Article I, Section 3 assigns to the Senate the power to try the impeachment, and requires a two-thirds vote to convict.\footnote{Article I, Section 3 (``The Senate shall have the sole power to try all impeachments. When sitting for that purpose, they shall be on oath or affirmation ...  And no person shall be convicted without the concurrence of two thirds of the members present.'').}
If the President of the United States is being tried, the Chief Justice of the United States Supreme Court must preside over the trial.\footnote{Article I, Section 3 (''When the President of the United States is tried, the Chief Justice shall preside'').}

The impeachment and conviction process is limited solely to removing or disqualifying someone from office.\footnote{Article I, Section 3 (``Judgment in cases of impeachment shall not extend further than to removal from office, and disqualification to hold and enjoy any office of honor, trust or profit under the United States'').}
Congress does not have the power to convict a person of a crime under its impeachment power.  However, if a person is impeached and convicted by Congress, the person removed from office is still subject to indictment, trial, and conviction in a court of law.\footnote{Article I, Section 3 (``the party convicted shall nevertheless be liable and subject to indictment, trial, judgment and punishment, according to law.'').}

\section{Restrictions on the Power of Congress}

Beyond just restricting the Federal government as a government of limited powers, the Constitution places further \textit{explicit} restrictions on the exercise of power by the Federal government.

\subsection{Restrictions Within Original Text}
There are a number of restrictions within the original text of the Constitution.

For example, while Article I, Section 8 says that Congress has the power to raise and support armies, it also places an explicit restriction that no law appropriating money for that purpose can last  longer than two years.\footnote{Article I, Section 8 (``To raise and support armies, but no appropriation of money to that use shall be for a longer term than two years;''}

Further, Article I, Section 7 imposes the restriction that all revenue-raising laws must originate in the House of Representatives.\footnote{Article I, Section 7 (``All bills for raising revenue shall originate in the House of Representatives; but the Senate may propose or concur with amendments as on other Bills.'').}

Additionally, Article I, Section 9 has a further list of explicit restrictions on the Federal government's power:
\begin{itemize}
\item Habeas Corpus cannot be suspended except when the public safety requires it and then only in the cases of rebellion or invasion.\footnote{Article I, Section 9 (``The privilege of the writ of habeas corpus shall not be suspended, unless when in cases of rebellion or invasion the public safety may require it.'').}
\item Congress cannot pass a bill of attainder or an Ex Post Facto Law.\footnote{Article I, Section 9 (``No bill of attainder or ex post facto Law shall be passed.'').}
\item The Government can't spend any money except pursuant to law.\footnote{Article I, Section 9 (``No money shall be drawn from the treasury, but in consequence of appropriations made by law.'').}
\item The United States government cannot grant any titles of nobility.\footnote{Article I, Section 9 (``No title of nobility shall be granted by the United States'').}
\end{itemize}

\subsection{Bill of Rights}
Probably the biggest set of restrictions, however, is found in the first ten amendments to the Constitution, also known as the Bill of Rights.
However, beyond just restricting the government's power, the Bill of Rights have achieved a special place in the United States society.  The rights and protections contained within the Bill of Rights have become so ingrained in the culture of the United States that they have become practically proverbial and self-evident amongst its people.

For example, the first amendment contains the restrictions that are colloquially known as ``freedom of speech,'' ``freedom of religion,'' and ``freedom of the press.''\footnote{First Amendment (``Congress shall make no law respecting an establishment of religion, or prohibiting the free exercise thereof; or abridging the freedom of speech, or of the press; or the right of the people peaceably to assemble, and to petition the government for a redress of grievances.'').}
The Fifth Amendment provides restrictions against Double Jeopardy (the government can't prosecute you twice for the same crime) and self-incrimination (known idiomatically as ``Taking the Fifth'' - the government cannot force you to incriminate yourself).\footnote{Fifth Amendment (''No person shall be held to answer for a capital, or otherwise infamous crime, unless on a presentment or indictment of a grand jury, except in cases arising in the land or naval forces, or in the militia, when in actual service in time of war or public danger; nor shall any person be subject for the same offense to be twice put in jeopardy of life or limb; nor shall be compelled in any criminal case to be a witness against himself, nor be deprived of life, liberty, or property, without due process of law; nor shall private property be taken for public use, without just compensation.'').}
The Sixth Amendment provides the right to a jury trial and assistance of counsel in criminal cases.\footnote{Sixth Amendment {``In all criminal prosecutions, the accused shall enjoy the right to a speedy and public trial, by an impartial jury of the State and district wherein the crime shall have been committed, which district shall have been previously ascertained by law, and to be informed of the nature and cause of the accusation; to be confronted with the witnesses against him; to have compulsory process for obtaining witnesses in his favor, and to have the Assistance of Counsel for his defence.''}}
The Eighth Amendment restricts the United States government from imposing cruel and unusual punishments or excessive bail or fines.\footnote{Eighth Amendment (``Excessive bail shall not be required, nor excessive fines imposed, nor cruel and unusual punishments inflicted.'').}

These are just several examples of the numerous restrictions on the government contained in the Bill of Rights.

\section{Congressional Action}
So, how does Congress act?  How does Congress exercise the legislative power of the United States?

\subsection{Quorum}
Before either house can take action, it must have a quorum of its members present to conduct business.
Article I, Section 5 of the Constitution requires that a majority of a house is requisite for it have a quorum.\footnote{Article I, Section 5 (``Each House shall be the judge of the elections, returns and qualifications of its own members, and a majority of each shall constitute a quorum to do business;'').}
The Constitution does make provision, however, for a smaller number to meet and, if necessary, to compel the attendance of absent members.\footnote{Article I, Section 5 (``but a smaller number may adjourn from day to day, and may be authorized to compel the attendance of absent members, in such manner, and under such penalties as each House may provide.'').}
Once a house has a sufficient number of its members present to constitute a quorum, it may take official action.

\subsection{Passing Both Houses}
So how does Congress take action?  The process of a bill becoming law follows the following general process:

\begin{itemize}
\item Introduction in one house
\item Committee Action
\item Report to house as a whole
\item Passage by that house
\item Introduction in other house.
\item Committee Action
\item Report to house as a whole
\item Passage by that house
\item Conference Committee
\end{itemize}

\subsection{Introduction In One House}
Every act of the Legislature begins its life as a proposed form of congressional action in one of the two houses.
There are four forms of congressional action: (1) Bill,  (2) Joint Resolution, (3) Concurrent Resolution, and (4) Simple Resolution.  For Congress to begin the process of taking action, one of these four types of proposals must be introduced in either the House or the Senate.

\subsubsection{Bill}
The bill is the most common form of congressional action.  
Most laws begin their lives as a bill in one house or the other.
A bill that passes both houses of congress is presented to the president to be signed into law.

\subsubsection{Joint Resolution}
A Joint Resolution is very similar to a bill.
There are some procedural differences, but practically there is little difference between the two.
A Joint Resolution that passes both houses of congress is presented to the president to be signed into law.

\subsubsection{Concurrent Resolution}
A concurrent resolution is an act of congress that makes no binding policy outside of congress and is not legislative in character.  
Like a bill or joint resolution, a concurrent resolution is intended to be passed by both houses.
However, unlike a bill or joint resolution, a concurrent resolution will not be presented to the President to be signed into law and will have no effect outside of Congress.
A concurrent resolution is, as described by the Supreme Court, ``a means of expressing fact, principles, opinions, and purposes of the two Houses, and thus does not need to be presented to the President.''\footnote{\textit{Bowsher v. Synar}, 478 U.S. 714, 756 (1986)(``A concurrent resolution, in contrast, makes no binding policy; it is a means of expressing fact, principles, opinions, and purposes of the two Houses, and thus does not need to be presented to the President.'')(internal citations and quotation marks omitted).}
Examples of concurrent resolutions include: 
\begin{itemize}
\item A concurrent resolution commending the National Association for the Advancement of Colored People on the occasion of its 102nd anniversary.
\item A concurrent resolution honoring the service and sacrifice of members of the United States Armed Forces who are serving in, or have served in, Operation Enduring Freedom, Operation Iraqi Freedom, and Operation New Dawn.
\item  A concurrent resolution authorizing the use of the rotunda of the Capitol for an event marking the 50th anniversary of the inaugural address of President John F. Kennedy.
\end{itemize}

\subsubsection{Simple Resolution}
Simple resolutions are similar to joint resolutions, but are meant to affect only one House of Congress.
Like a Joint Resolution, a Simple Resolution is a nonbinding, non-legislative act that has no effect outside of congress.
However, Simple Resolutions affect only the house of congress that passes it.
For example, the rules of the House of Representative are passed via a Simple Resolution.
Other examples of a simple resolution:
\begin{itemize}
\item A resolution designating a particular day as ``Small Business Saturday'' and supporting efforts to increase awareness of the value of locally owned small businesses.
\item  A resolution recognizing National Native American Heritage Month and celebrating the heritages and cultures of Native Americans and the contributions of Native Americans to the United States.
\end{itemize}

\subsection{Committee Action}

\subsubsection{Referral}
Once a bill is introduced in a house it is referred to committee for consideration and markup.
A committee is a subgroup within a house that has jurisdiction and initial control over all bills that fall within the purview of the committee's assigned subject matters.

For example, the Senate has the following standing committees:

\begin{itemize}
\item Agriculture, Nutrition, and Forestry 
\item Appropriations 
\item Armed Services 
\item Banking, Housing, and Urban Affairs 
\item Budget 
\item Commerce, Science, and Transportation 
\item Energy and Natural Resources 
\item Environment and Public Works 
\item Finance 
\item Foreign Relations 
\item Health, Education, Labor, and Pensions 
\item Homeland Security and Governmental Affairs 
\item Judiciary 
\item Rules and Administration 
\item Small Business and Entrepreneurship 
\item Veterans' Affairs 
\end{itemize}

Each Committee has its subject matter jurisdiction set by rule of that house.
For example, in the House of Representatives, the Committee on the Judiciary has jurisdiction over the following areas\footnote{RULE X - ORGANIZATION OF COMMITTEES
Committees and their legislative jurisdictions
1. There shall be in the House the following standing committees, each of which shall have the jurisdiction and related functions assigned by this clause and clauses 2, 3, and 4. All bills, resolutions, and other matters relating to subjects within the jurisdiction of the standing committees listed in this clause shall be referred to those committees, in accordance with clause 2 of rule XII, as follows:
(l) Committee on the Judiciary. (1) The judiciary and judicial proceedings, civil and criminal. (2) Administrative practice and procedure. (3) Apportionment of Representatives. (4) Bankruptcy, mutiny, espionage, and counterfeiting. (5) Civil liberties. (6) Constitutional amendments. (7) Criminal law enforcement. (8) Federal courts and judges, and local courts in the Territories and possessions. (9) Immigration policy and non-border enforcement. (10) Interstate compacts generally. (11) Claims against the United States. (12) Meetings of Congress; attendance of Members, Delegates, and the Resident Commissioner; and their acceptance of incompatible offices. (13) National penitentiaries. (14) Patents, the Patent and Trademark Office, copyrights, and trademarks. (15) Presidential succession. (16) Protection of trade and commerce against unlawful restraints and monopolies. (17) Revision and codification of the Statutes of the United States. (18) State and territorial boundary lines. (19) Subversive activities affecting the internal security of the United States.}:

\begin{enumerate}
\item The judiciary and judicial proceedings, civil and criminal. 
\item Administrative practice and procedure. 
\item Apportionment of Representatives. 
\item Bankruptcy, mutiny, espionage, and counterfeiting. 
\item Civil liberties. 
\item Constitutional amendments. 
\item Criminal law enforcement. 
\item Federal courts and judges, and local courts in the Territories and possessions. 
\item Immigration policy and non-border enforcement. 
\item Interstate compacts generally. 
\item Claims against the United States. 
\item Meetings of Congress; attendance of Members, Delegates, and the Resident Commissioner; and their acceptance of incompatible offices. 
\item National penitentiaries. 
\item Patents, the Patent and Trademark Office, copyrights, and trademarks. 
\item Presidential succession. 
\item Protection of trade and commerce against unlawful restraints and monopolies. 
\item Revision and codification of the Statutes of the United States. 
\item State and territorial boundary lines. 
\item Subversive activities affecting the internal security of the United States.
\end{enumerate}

When a bill is introduced that falls within a particular committee's assigned subject matter, the bill gets referred to that committee.
For example, if a bill were introduced in the House of Representatives to change bankruptcy law, that bill would be referred to the Committee on the Judiciary, since bankruptcy is listed in the subjects that fall under the Committee on the Judiciary's jurisdiction.

\subsubsection{Review and Markup}
Once the bill gets to the committee, the committee performs a review and markup process.
This usually involves investigating the bill's subject area and seeking input from other relevant areas of the federal government, such as the executive branch or the appropriate agency or department, as well as the Government Accountability Office for an opinion about the necessity or propriety of enacting the proposed bill.
The committee will then often times hold hearings to allow committee members to ask questions of and take testimony from witnesses.  Committees can even issue subpoenas to compel the attendance and testimony of a witness.
After the hearings, the committee will go through the markup process with the bill, where the committee makes changes to the bill that it considers appropriate.  After the committee has finished marking up the bill it will then report the bill with the committee's proposed changes and recommendations to the house at large.

\subsection{Calendars}
Once a bill is reported to the house at large, it is put on one of that house's calendars for consideration by the entire house.

\subsubsection{House Calendars}
In the House of Representatives, there are four calendars\footnote{Rules of House of Representatives, RULE XIII - CALENDARS AND COMMITTEE REPORTS Calendars:  1. (a) All business reported by committees shall be referred to one of the following three calendars: (1) A Calendar of the Committee of the Whole House on the state of the Union, to which shall be referred public bills and public resolutions raising revenue, involving a tax or charge on the people, directly or indirectly making appropriations of money or property or requiring such appropriations to be made, authorizing payments out of appropriations already made, releasing any liability to the United States for money or property, or referring a claim to the Court of Claims. (2) A House Calendar, to which shall be referred all public bills and public resolutions not requiring referral to the Calendar of the Committee of the Whole House on the state of the Union. (3) A Private Calendar as provided in clause 5 of rule XV, to which shall be referred all private bills and private resolutions. (b) There is established a Calendar of Motions to Discharge Committees as provided in clause 2 of rule XV.}:

\begin{enumerate}
\item \textbf{The Committee of the Whole House on the state of the Union (aka Union Calendar)}.  This is the calendar for all public bills and resolutions raising revenue, involving a tax or charge on the people, directly or indirectly making appropriations of money or property or requiring such appropriations to be made, authorizing payments out of appropriations already made, releasing any liability to the United States for money or property, or 
referring a claim to the Court of Claims.\footnote{Rules of House of Representatives RULE XIII 1. (a) (1) A Calendar of the Committee of the Whole House on the state of the Union, to which shall be referred public bills and public resolutions raising revenue, involving a tax or charge on the people, directly or indirectly making appropriations of money or property or requiring such appropriations to be made, authorizing payments out of appropriations already made, releasing any liability to the United States for money or property, or referring a claim to the Court of Claims.}
This is the general calendar for the majority of public bills and resolutions in the House of Representatives
\item \textbf{House Calendar}.  This is the calendar for all public bills and resolutions not requiring referral to the Calendar of the Committee of the Whole House on the state of the Union.\footnote{Rules of House of Representatives, RULE XIII 1 (a) (2) A House Calendar, to which shall be referred all public bills and public resolutions not requiring referral to the Calendar of the Committee of the Whole House on the state of the Union.}  Meaning, all public bills and resolutions not referred to the Union Calendar are referred to the House Calendar.

\item \textbf{Private Calendar}.  This is the calendar for all private bills and resolutions.\footnote{Rules of House of Representatives, RULE XIII 1 (a) (3) A Private Calendar as provided in clause 5 of rule XV, to which shall be referred all private bills and private resolutions.}
\item \textbf{Calendar of Motions to Discharge Committees}.  This calendar serves a special purpose for motions to discharge committees, which is a procedural request to expedite consideration of pending bills or resolutions still in committee.\footnote{Rules of House of Representatives, RULE XIII 1 (b) There is established a Calendar of Motions to Discharge Committees as provided in clause 2 of rule XV.}

\end{enumerate}


\subsubsection{Senate Calendars}
While the House of Representatives has four calendars, in the Senate there are only two Calendars:

\begin{enumerate}
\item \textbf{Calendar of Business}.  This is the calendar for all legislation to be considered by the Senate.
\item \textbf{Executive Calendar}. This is the calendar for executive resolutions, treaties, and nominations.
\end{enumerate}

\subsection{Voting}
When a legislative act has been called on a house's calendar, the house votes on whether to accept or reject the bill.  There are often further procedural steps that aren't detailed here that a bill goes through between when it is placed on the calendar and when it is called up for a vote.  However, after the bill has traversed the procedural steps, the end result is that a vote is called on the bill and the house votes on the bill.

\subsubsection{Voting in the House of Representatives}

The House of Representatives has several methods for taking a vote\footnote{Rule XX - Voting And Quorum Calls
1. (a) The House shall divide after the Speaker has put a question to a vote by voice as provided in clause 6 of rule I if the Speaker is in doubt or division is demanded. Those in favor of the question shall first rise from their seats to be counted, and then those opposed. 
(b) If a Member, Delegate, or Resident Commissioner requests a recorded vote, and that request is supported by at least one-fifth of a quorum, the vote shall be taken by electronic device unless the Speaker invokes another procedure for recording votes provided in this rule. A recorded vote taken in the House under this paragraph shall be considered a vote by the yeas and nays. 
(c) In case of a tie vote, a question shall be lost.}:

\begin{itemize}
\item \textbf{Voice Vote}.  The first method of taking a vote in the House of Representatives is a voice vote.  When the speaker calls for a voice vote, all those in favor of passing the legislative action called for a vote say aloud ``Aye''.  Those opposed say ``Nay.''  The winner is determined by comparing the number of Ayes to the number of Nays.
\item \textbf{Division}.  The second method of taking a vote is by ``division.''  After a voice vote has been taken, if the speaker of the house is in doubt as to the result or if a division vote is requested, then a vote by ``division'' will be taken.  When called to vote, those in favor stand up and are counted, and then afterwards those opposed stand and are counted.\footnote{Rule XX 1 (a) The House shall divide after the Speaker has put a question to a vote by voice as provided in clause 6 of rule I if the Speaker is in doubt or division is demanded. Those in favor of the question shall first rise from their seats to be counted, and then those opposed.}
\item \textbf{Record Vote}.  The third method of taking a vote in the House of Representatives is by a record vote.  If a member of the House of Representatives requests a recorded vote and is supported by at least 1/5 of a quorum, the vote is taken by electronic device.\footnote{Rule XX 1 (b) If a Member, Delegate, or Resident Commissioner requests a recorded vote, and that request is supported by at least one-fifth of a quorum, the vote shall be taken by electronic device unless the Speaker invokes another procedure for recording votes provided in this rule. A recorded vote taken in the House under this paragraph shall be considered a vote by the yeas and nays.}
\end{itemize}

To pass, a vote needs a simple majority, except when the vote concerns an act which includes an increase in federal income tax, which then requires passage by a three-fifths vote.\footnote{Rule XX (c) In case of a tie vote, a question shall be lost.}\footnote{Rule XXI - Restrictions On Certain Bills.  Passage of tax rate increases. (b) A bill or joint resolution, amendment, or conference report carrying a Federal income tax rate increase may not be considered as passed or agreed to unless so determined by a vote of not less than three-fifths of the Members voting, a quorum being present.}
In the case of a tie vote, the issue is considered to not have passed.\footnote{Rule XX (c) In case of a tie vote, a question shall be lost.}


\subsubsection{Voting in the Senate}

There are three methods of voting in the Senate:

\begin{itemize}
\item \textbf{Voice Vote}.  The first method is a voice vote.  Like the House's voice vote, when the Senate's presiding office calls a voice vote, the yea's announce their vote in favor and the nays then announce their vote against.  The presiding officer determines the winner by comparing the number in favor to the number against. 
\item \textbf{Division}.  The second method is a vote by division.  Also like the House, the Senate can take a vote by division where those in favor stand up and are counted, and then afterwards those opposed stand and are counted.
\item \textbf{Roll Call Vote}.  The third method is a roll call vote.  In a roll call vote, the names of the Senators are called alphabetically and each Senator declares his or her assent or dissent to the issue being voted on.\footnote{Senate Rule XII VOTING PROCEDURE. When the yeas and nays are ordered, the names of Senators shall be called alphabetically; and each Senator shall, without debate, declare his assent or dissent to the question, unless excused by the Senate; and no Senator shall be permitted to vote after the decision shall have been announced by the Presiding Officer, but may for sufficient reasons, with unanimous consent, change or withdraw his vote. No motion to suspend this rule shall be in order, nor shall the Presiding Officer entertain any request to suspend it by unanimous consent.}
\end{itemize}

The winner of a vote in the Senate is determined by a simple majority.  When there is a tie, the Constitution allows the Vice President, as the President of the Senate, to cast a tie-breaker vote.\footnote{Article I, Section 3 (``The Vice President of the United States shall be President of the Senate, but shall have no Vote, unless they be equally divided.'').}

\subsubsection{Filibuster}
Like the House of Representatives, the Senate Rules allow for debate prior to voting on a bill or resolution.  However, unlike the House of Representatives, there is no set time limit for the debate in the Senate.  Rather, the Senate rules implicitly allow for debate to continue indefinitely so long as a Senator wishes to keep it going.  Thus, if a Senator wants to block a bill or resolution from ever being voted on, the Senator need only continue debate on the bill and not yield the floor.  Per the Senate rules, in order to close off debate in the Senate a cloture motion must be made and passed.
Per the rules, a cloture motion must be made by at least 16 Senators and it must pass by a vote of at least 60 Senators (that is, three-fifths of the Senate).\footnote{See Senate Rule XXII}
If the proponents of a cloture motion cannot garner 60 votes to bring the debate to a close, then debate on the bill or resolution can continue indefinitely, and it will never come to a vote.  This process of holding up a vote by endlessly continuing debate is known as the ``filibuster''.  A filibuster is a particularly effective tool for a Senator who wants to defeat a bill or resolution from passing but does not have the necessary votes to vote it down.  Instead, by keeping the debate from ever ending, one Senator can indefinitely stall the voting process and prevent the bill or resolution from ever coming to a vote.  Practically speaking, this has the same effect as voting the bill or resolution down because the bill will never come to a vote so it will never pass the Senate.

\subsubsection{After Passing One House}
Once a bill or resolution passes one of the houses it is known as an ``act.''  

\subsection{Conference Committee}
Once a bill passes one house, it must also pass the other house in the exact same form before it can become law.  Often times, however, a bill or resolution will pass each house in different forms.  Thus, once a bill has passed both houses, it must be reconciled to resolve any differences between the two different versions.
Congress does this by convening a Conference Committee made up of members of both Houses.
Each house sends representatives to the Conference Committee to reconcile the two versions.
Once the Conference Committee has resolve the differences between the two versions of the bill, the reconciled bill is returned to both houses and the conformed version must pass both houses in identical form.
After the bill has passed both houses in identical form, the bill or resolution is considered passed by Congress and is then sent to the President for his signature.

