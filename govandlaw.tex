% ******************************************************** %
% Copyright © Jonathan Gitlin 2011 – 2013
%
% This work is licensed under the Creative Commons 
% Attribution-NonCommercial-ShareAlike 3.0 Unported License
%
% For More Information See the Creative Commons Website
% http://creativecommons.org
%
% For A human readable summary, See:
% http://creativecommons.org/licenses/by-nc-sa/3.0/
%
% For the Full Text of the License:
% http://creativecommons.org/licenses/by-nc-sa/3.0/legalcode
%
% No Claim Is Made To Original Works 
% Of Other Sources Cited Within This Work
%
% ******************************************************** %

\chapter{Government and Law}

\section{Purpose of Chapter}

The purpose of this chapter is to cover (1) the basic structure of the United States government and (2) how the United States government creates its laws.

\section{The United States Government}

Every government must begin somewhere, with some basis for its legitimacy.  For the United States, the Constitution is that basis.  All authority for the United States government begins at and emanates from the Constitution.  The Constitution gives the United States government its power, establishes the basic structure of that government, and defines its principal boundaries and functions.


\section{The Constitution}

\subsection{Written Document}
The United States Constitution is contained within a single written document.  And while this may be taken for granted today, it was not always the case.  Under British rule prior to the American Revolution the colonies had no single written constitution.  Instead, they had the English Constitution, which was a mish-mash collection of legislative acts, ancient historical documents (such as the Magna Carta), common law traditions, legal precedents, and core concepts that had achieved constitutional status over the years in English society.  However, because there was no unified collection that compiled all of the constituent parts into a single writing it was difficult to ascertain with any certainty the precise boundaries and contents of the British Constitution.  Rather, the British constitution existed as an amorphous anthology.

In contrast, the United State Constitution is a single written document whose contents are definite and easy to ascertain.

\subsection{Structure of the Constitution}
The United States Constitution is divided into 7 articles.  Article I establishes the legislative branch, Article II the executive, and Article III the judicial.  Article IV deals primarily with states.  Article V deals with amendments to the constitution.  Article VI establishes the supremacy of the United States laws.  Article VII is the bootstrap provision -- establishing the number of ratifying state constitutional conventions necessary to ``start'' the United States government under the Constitution.

\subsection{Amendments to the Constitution}
\subsubsection{Amendment Process}
The Constitution contemplates that it will be amended from time to time, and Article V lays out the process for doing this:

\begin{quote}
The Congress, whenever two thirds of both Houses shall deem it necessary, shall propose Amendments to this Constitution, or, on the Application of the Legislatures of two thirds of the several States, shall call a Convention for proposing Amendments, which, in either Case, shall be valid to all Intents and Purposes, as Part of this Constitution, when ratified by the Legislatures of three fourths of the several States or by Conventions in three fourths thereof, as the one or the other Mode of Ratification may be proposed by the Congress
\end{quote}

\subsubsection{Amendment Proposal}
As laid out in the text above, Article V provides two possible ways to propose amendments to the Constitution:
\begin{enumerate}
\item \textbf{Congressional Proposal} When two thirds of both houses of Congress vote to propose an amendment; or
\item \textbf{State Proposal} When two thirds of the state legislatures apply for a convention to propose amendments.
\end{enumerate}

\subsubsection{Amendment Ratification}
Once an amendment has been proposed, Article V provides for two ways for it to be ratified:
\begin{enumerate}
\item \textbf{State Approval} If three fourths of the state legislatures approve the amendment; or
\item \textbf{Convention Approval} If conventions in three fourths of the states approve the amendment.
\end{enumerate}


\subsubsection{The Various Amendments}

The Constitution has been amended a number of times since its adoption:

\begin{itemize}
\item The first 10 amendments are referred to as the ``Bill of Rights.'' They detail certain protections for citizens of the United States against the federal government, such as freedom of speech and press, guarantees of due process and speedy trials, and prohibitions against unreasonable searches and seizures.
\item The 11th Amendment modified the power of federal courts to respect State sovereign immunity.
\item The 12th Amendment created Presidential tickets in presidential elections.\footnote{See the chapter on the Executive for a more detailed discussion of this amendment.}
\item The 13th Amendment abolished slavery in the United States.
\item The 14th Amendment is a multifaceted amendment that, among other things, fundamentally changed the power of State governments and established birthright citizenship.
\item The 15th Amendment established racial equality in voting rights.
\item The 16th Amendment gave Congress the power to impose income taxes.
\item The 17th Amendment changed the election of Senators.  Under the original method, Senators were elected by state legislatures.  Under this amendment, Senators were elected by state-wide popular vote.
\item The 18th Amendment established Prohibition; that is, it outlawed alcohol.
\item The 19th Amendment established gender equality in voting rights. 
\item The 20th Amendment dealt with, among other things, the terms of the President and Congress.
\item The 21st Amendment repealed Prohibition.
\item The 22nd Amendment imposed a two-term limit on the President.
\item The 23rd Amendment gave the District of Columbia electors in presidential elections.
\item The 24th Amendment protected voting rights from infringement by poll taxes.
\item The 25th Amendment deals with incapacity or vacancy in the office of the President or Vice President.
\item The 26th Amendment established the voting age as 18 years old.
\item The 27th Amendment  limits when Congress can give itself a pay raise.
\end{itemize}

\subsubsection{Proposed Amendments That Did Not Pass}
There have been several amendments that Congress proposed that were never ratified by the States.  Two examples of such amendments are (1) the Equal Rights Amendment and (2) the originally proposed 13th Amendment.


\begin{itemize}
\item \textbf{The Equal Rights Amendment}
Section 1 of the 14th Amendment contains the Equal Protection clause, which provides that no State shall deny to any person the equal protection of the laws.  

\begin{quote}
No State shall make or enforce any law which shall abridge the privileges or immunities of citizens of the United States; nor shall any State deprive any person of life, liberty, or property, without due process of law; nor deny to any person within its jurisdiction the equal protection of the laws.
\end{quote}

The Supreme Court has found that this provision provides strong protection against laws that discriminate on the basis of race, alienage, or national origin.\footnote{\textit{Cleburne v. Cleburne Living Center, Inc.}, 473 U.S. 432, 440 (1985)(``The general rule is that legislation is presumed to be valid and will be sustained if the classification drawn by the statute is rationally related to a legitimate state interest ... The general rule gives way, however, when a statute classifies by race, alienage, or national origin ... For these reasons and because such discrimination is unlikely to be soon rectified by legislative means, these laws are subjected to strict scrutiny and will be sustained only if they are suitably tailored to serve a compelling state interest.'').}
However, the Supreme Court has \textit{not} interpreted the Equal Protection clause to provide the same kind of protection against gender-based discrimination.  Rather, the Court has interpreted the Equal Protection clause of the 14th Amendment to provide weaker protection against laws that discriminate on the basis of gender.\footnote{\textit{Cleburne v. Cleburne Living Center, Inc.}, 473 U.S. 432, 440 (1985)(``Legislative classifications based on gender also call for a heightened standard of review ... A gender classification fails unless it is substantially related to a sufficiently important governmental interest.'').}

Therefore, in order to equalize the protection afforded under the Constitution for gender-based discrimination with other factors such as race and national origin, proponents drafted and proposed a constitutional amendment to affect this equalization.  The proposed amendment expressly prohibited gender-based discrimination under the law; the key text of the ERA said ``Equality of rights under the law shall not be denied or abridged by the United States or by any State on account of sex.''\footnote{United States Government Printing Office, ``Proposed Amendments Not Ratified By The States''.}
While the proposed Equal Rights Amendment has come up for consideration a number of times in the 20th Century, it has never been passed by both Congress and the States.  The closest it got was when the proposed amendment passed both houses of Congress in the 1970's but it fell short of the necessary State approval by 3 states.

\item \textbf{Original 13th Amendment}
The 13th Amendment, which was passed by Congress on January 31, 1865 at the end of the Civil War and ratified by the States on December 6, 1865 after the Civil War was concluded, abolished slavery in the United States.\footnote{Thirteenth Amendment, Section 1 (``Neither slavery nor involuntary servitude, except as a punishment for crime whereof the party shall have been duly convicted, shall exist within the United States, or any place subject to their jurisdiction.'').}  However, Congress had previously attempted to prevent itself from doing just this.  On March 2, 1861, just before Abraham Lincoln took office and approximately one month prior to the Battle of Fort Sumter that initiated the Civil War, Congress had passed a constitutional amendment that prohibited the further amendment of the constitution to allow Congress to outlaw slavery.  The substantive text of the proposed amendment read:

\begin{quote}
No amendment shall be made to the Constitution which will authorize or give to Congress the power to abolish or interfere, within any State, with the domestic institutions thereof, including that of persons held to labor or service by the laws of said State.\footnote{United States Government Printing Office, ``Proposed Amendments Not Ratified By The States''.}
\end{quote}

Although not required, President James Buchanan signed the amendment after Congress passed it and before it was submitted to the States.\footnote{United States Government Printing Office, ``Proposed Amendments Not Ratified By The States''.}  This is the only proposed amendment to the Constitution to ever be signed by a sitting president.\footnote{United States Government Printing Office, ``Proposed Amendments Not Ratified By The States'', (``this is the only proposed amendment to the Constitution ever signed by the President.'').}  However, despite the proposed amendment passing both houses of Congress by the required number of votes and the President's signature, the proposed amendment was never ratified by the States.  
\end{itemize}

\subsubsection{Effect of Amendment}

On June 8, 1789, James Madison submitted to Congress a bill proposing the changes to the Constitution that would go on to become the first ten amendments to the Constitution, what we know today as the Bill of Rights.\footnote{Annals of Congress, House of Representatives, 1st Congress, 1st Session, June 8, 1789, p. 440 -- 41, Washington, DC:  Gales and Seaton, 1834.}
However, during the debate on these amendments, a disagreement arose as to what effect the amendments would have on the form of the Constitution.

Madison wanted the amendments to alter the original language of the Constitution, leaving the Constitution with different text after the amendment.  For example, Madison's bill proposed the following changes:

\begin{quote}
That in article 1st, section 2, clause 3, these words be struck out, to wit: ``The number of Representatives shall not exceed one for every thirty thousand, but each State shall have at least one Representative, and until such enumeration shall be made;'' and that in place thereof be inserted these words, to wit: ``After the first actual enumeration, there shall be one Representative for every thirty thousand, until the number amounts to ---,  after which the proportion shall be so regulated by Congress, that the  number shall never be less than ---, nor more than ---, but each State shall, after the first enumeration, have at least two Representatives; and prior thereto.'' ...

That in article 1st, section 9, between clauses 3 and 4, be inserted these clauses, to wit: The civil rights of none shall be abridged on account of religious belief or worship, nor shall any national religion be established, nor shall the full and equal rights of conscience be in any manner, or on any pretext, infringed.
\end{quote}

However, others in Congress disagreed with Madison's view as to what effect an amendment should have on the Constitution's original language.  They believed that rather than altering the original text of the Constitution that any amendments should instead be appended to the \textit{end} of the Constitution, leaving the original text unmodified.  During the debate on this issue, Georgia Congressman James Jackson opined that

\begin{quote}
the original constitution ought to remain inviolate, and not be patched up, from time to time, with various stuffs resembling Joseph's coat of many colors.  Some gentlemen talk of repealing the present constitution, and adopting an improved one. If we have this power, we may go on from year to year, making new ones; and in this way, we shall render the basis of the superstructure the most fluctuating thing imaginable, and the people will never know what the constitution is.\footnote{Annals of Congress, House of Representatives, 1st Congress, 1st Session, August 13, 1789, p. 741, Washington, DC:  Gales and Seaton, 1834 (``Mr Jackson. -- I do not like to differ with gentlemen about form; but as so much has been said, I wish to give my opinion; it is this: that the original constitution ought to remain inviolate, and not be patched up, from time to time, with various stuffs resembling Joseph's coat of many colors.
Some gentlemen talk of repealing the present constitution, and adopting an improved one. If we have this power, we may go on from year to year, making new ones; and in this way, we shall render the basis of the superstructure the most fluctuating thing imaginable, and the people will never know what the constitution is.'').}
\end{quote}

Madison worried, though, that placing the amendments at the end would lead to ambiguity because he believed it would be difficult to tell to what extent the original text was altered by the amendment.

\begin{quote}
if they are supplementary, its meaning can only be ascertained by a comparison of the two instruments, which will be a very considerable embarrassment. It will be difficult to ascertain to what parts of the instrument the amendments particularly refer; they will create unfavorable comparisons; whereas, if they are placed upon the footing here proposed, they will stand upon as good foundation as the original work.\footnote{Annals of Congress, House of Representatives, 1st Congress, 1st Session, August 13, 1789, p. 736, Washington, DC:  Gales and Seaton, 1834. (`` Now it appears to me, that there is a neatness and propriety in incorporating the amendments into the Constitution itself; in that case, the system will remain uniform and entire; it will certainly be more simple when the amendments are interwoven into those parts to which they naturally belong, than it will if they consist of separate and distinct parts. We shall then be able to determine its meaning without references or comparison; whereas, if they are supplementary, its meaning can only be ascertained by a comparison of the two instruments, which will be a very considerable embarrassment. It will be difficult to ascertain to what parts of the instrument the amendments particularly refer; they will create unfavorable comparisons; whereas, if they are placed upon the footing here proposed, they will stand upon as good foundation as the original work. Nor is it so uncommon a thing as gentlemen suppose; systematic men frequently take up the whole law, and, with its amendments and alterations, reduce it into one act.'').}  
\end{quote}

Despite Madison's preference for internal modification, though, those who favored appending the amendments to the end of the Constitution won the debate.  Thus, after the first amendments were approved by the states they were added to the end of the Constitution as opposed to altering the original text, and this established the precedent for all future amendments. 

Therefore, to read the constitution properly, the original text must always be read within the context of all subsequent amendments.  For example: in the original text of the constitution, Article I, Section 3 provides that Senators are to be chosen by their respective State legislatures.  However, the 17th Amendment says that Senators are to be chosen by popular election in their respective States.  Thus, the 17th Amendment alters the manner in which Senators are elected.  However, the original provision in Article I Section 3 remains unchanged and is still a ``part'' of the Constitution.  Thus, the original text and the amendments must be read together as a whole to determine the functional provisions of the Constitution.

This even applies to the amendments themselves.  For example, the 18th Amendment established Prohibition, and the 21st Amendment repealed it.\footnote{18th Amendment, Section 1 (``After one year from the ratification of this article the manufacture, sale, or transportation of intoxicating liquors within, the importation thereof into, or the exportation thereof from the United States and all territory subject to the jurisdiction thereof for beverage purposes is hereby prohibited.'');  21st Amendment, Section 1 (``The eighteenth article of amendment to the Constitution of the United States is hereby repealed.'').}  However, despite the fact that the 21st amendment nullified the 18th Amendment, the 18th Amendment is still a ``part'' of the Constitution -- just a part that has no effect.

Thus, the original text and the amendments must be read all together to properly ascertain the contents and provisions of the Constitution.

\section{Structure of Government \& Separation of Powers}

The Constitution divides the United States government into three distinct branches -- the Legislative Branch (the branch that makes the laws), the Executive Branch (the branch that executes and enforces the laws), and the Judicial Branch (the court system).  The purpose of separating the power of the United States government into separate branches is to prevent the consolidation of power, which many in the post-Revolutionary War era saw as a direct threat to liberty.  In fact, James Madison wrote in the Federalist \#47 that the consolidation of all powers, legislative, executive, and judicial, could ``justly be pronounced the very definition of tyranny.''

\begin{quote}
The accumulation of all powers, legislative, executive, and judiciary, in the same hands, whether of one, a few, or many, and whether hereditary, selfappointed, or elective, may justly be pronounced the very definition of tyranny.\footnote{Jame Madison, Federalist \# 47}
\end{quote}

In order to protect against tyranny, Madison said, the United States government's powers should be split amongst co-equal branches.  Further, each of the separate branches should be given the power and motive to resist encroachments by the other branches.  With each branch having an independent ability and motive to resist consolidation of power by the other branches, each branch was to act as a check and balance on the other two.  Or, as Madison succinctly put it in the Federalist \#51, ``Ambition must be made to counteract ambition.''

\begin{quote}
But the great security against a gradual concentration of the several powers in the same department, consists in giving to those who administer each department the necessary constitutional means and personal motives to resist encroachments of the others... Ambition must be made to counteract ambition.\footnote{James Madison, Federalist \#51}
\end{quote}

Thus the Constitution separates the Legislative, Executive, and Judicial powers of the United States into three distinct branches.  However, though it is axiomatic that the separation of powers is inherent in the United States Government, nowhere in the Constitution does the phrase ``separation of powers'' appear.  Rather, it is inferred by the structure of the government and the writings about it by the various political thinkers of the era, including the drafters of the Constitution.

Interestingly, though, in the original text of the Bill of Rights submitted by Madison to Congress, Madison included an explicit ``separation of powers'' clause.  The proposed language read:
\begin{quote}
The powers delegated by this Constitution are appropriated to the departments to which they are respectively distributed: so that the Legislative Department shall never exercise the powers vested in the Executive or Judicial, nor the Executive exercise the powers vested in the Legislative or Judicial, nor the Judicial exercise the powers vested in the Legislative or Executive Departments.\footnote{Annals of Congress, House of Representatives, 1st Congress, 1st Session, June 8, 1789, p. 453, Washington, DC:  Gales and Seaton, 1834.}
\end{quote}

This language made it through committee in the House and passed the House as a whole, but was removed at some point during its consideration in the Senate.  Thus, there remains no explicit ``separation of powers'' language in the Constitution, but instead must be inferred from the rest of its text.

\section{How a Law is Passed}

All exercise of power by the United States government must be pursuant to law, and Article I, Section 7 of the Constitution details how the United States creates a law:

\begin{quote}
Every Bill which shall have passed the House of Representatives and the Senate, shall, before it become a Law, be presented to the President of the United States: If he approve he shall sign it, but if not he shall return it, with his Objections to that House in which it shall have originated, who shall enter the Objections at large on their Journal, and proceed to reconsider it. If after such Reconsideration two thirds of that House shall agree to pass the Bill, it shall be sent, together with the Objections, to the other House, by which it shall likewise be reconsidered, and if approved by two thirds of that House, it shall become a Law. But in all such Cases the Votes of both Houses shall be determined by yeas and Nays, and the Names of the Persons voting for and against the Bill shall be entered on the Journal of each House respectively. If any Bill shall not be returned by the President within ten Days (Sundays excepted) after it shall have been presented to him, the Same shall be a Law, in like Manner as if he had signed it, unless the Congress by their Adjournment prevent its Return, in which Case it shall not be a Law.
\end{quote}

Each part of this process is detailed below.

\subsubsection{Basic Bill Process}

Every law begins as a bill (or its functional equivalent), proposed in one of the houses of Congress.  The Constitution then requires that both houses of Congress must vote to approve the bill.\footnote{Article I, Section 7(``'Every Bill which shall have passed the House of Representatives and the Senate, shall, before it become a Law, be presented to the President of the United States'')}
Next, the bill is presented to the President for his approval.\footnote{Article I, Section 7(``Every Bill which shall have passed the House of Representatives and the Senate, shall, before it become a Law, be presented to the President of the United States''')}
If the president approves of the bill, he signs it and it becomes law.\footnote{Article I, Section 7(``If he approve he shall sign it''')}

\subsubsection{Veto Process}
However, if the President does not approve of the bill, he has the power to veto it, thus preventing it, at least temporarily, from becoming law.\footnote{Article I, Section 7 (``If he approve he shall sign it, but if not he shall return it, with his Objections to that House in which it shall have originated, who shall enter the Objections at large on their Journal, and proceed to reconsider it'')}
The power to prevent a bill that has passed both houses of Congress from becoming law gives the President a strong power in the law-making process of the United States government.

\subsubsection{Veto Time Limit}
There are, however, several restrictions on the President's veto power.  First, the President only has 10 days to exercise his veto power.\footnote{Article I, Section 7 (``If any Bill shall not be returned by the President within ten Days (Sundays excepted) after it shall have been presented to him, the Same shall be a Law, in like Manner as if he had signed it'')}  If the President neither signs nor vetoes a bill within this 10-day window, the President is deemed to have accepted the bill and it becomes a law, just as if he had signed it.\footnote{Article I, Section 7 (``If any Bill shall not be returned by the President within ten Days (Sundays excepted) after it shall have been presented to him, the Same shall be a Law, in like Manner as if he had signed it'')}

This deemed acceptance, however, also has a limitation.  If Congress adjourns during the 10 day veto period and the President has neither signed nor vetoed the bill by the time Congress adjourns, then the bill is deemed vetoed as if the President had actually done so.\footnote{Article I, Section 7 (``If any Bill shall not be returned by the President within ten Days (Sundays excepted) after it shall have been presented to him, the Same shall be a Law, in like Manner as if he had signed it, unless the Congress by their Adjournment prevent its Return, in which Case it shall not be a Law.'')}
This automatic veto is referred to as a ``Pocket Veto.''

\subsubsection{Congressional Veto Override}
The President's veto power is further limited by Congress' ability to override his veto.  Just as the President has the power to override Congress by vetoing a bill, Congress has the power to override the President's veto via a subsequent vote.\footnote{Article I, Section 7 (``If after such Reconsideration two thirds of that House shall agree to pass the Bill, it shall be sent, together with the Objections, to the other House, by which it shall likewise be reconsidered, and if approved by two thirds of that House, it shall become a Law.'')}
If, after the President vetoes a bill, both houses reapprove the bill by a two-thirds vote, then it becomes law notwithstanding the President's veto.\footnote{Article I, Section 7 (``If after such Reconsideration two thirds of that House shall agree to pass the Bill, it shall be sent, together with the Objections, to the other House, by which it shall likewise be reconsidered, and if approved by two thirds of that House, it shall become a Law.'')}

\subsubsection{Summary}

In summary, the Constitution establishes the following process for a bill to become a law in the United States:
\begin{enumerate}
\item A bill must pass both houses of Congress.
\item After passing both houses, the bill must be presented to the President for his approval. 
\item If the President approves of the bill he can sign it into law, or if he disapproves of it he can veto it.
\item If the President neither signs nor vetoes the law within 10 days, however, the President is deemed to have accepted the bill and it becomes law as if he had signed it, unless, within that 10 days, Congress adjourns and then the President is deemed to have vetoed the bill.
\item If, however, the President vetoes a bill when Congress is still in session, Congress has the power to override his veto by a two-thirds vote in both houses, and the bill then becomes law, the President's veto notwithstanding.
\end{enumerate}


\subsubsection{Line-Item Veto}
Practically speaking, bills that pass through Congress are often extensive in scope and size, and contain a vast number of provisions covering a wide range of legislative topics.  Accordingly, there are times when a President, when presented with such a bill for his approval, would prefer to only sign certain parts of the bill into law and to veto the other parts.  The concept of vetoing only parts of a bill is known as a ``line item veto'' -- the idea being that the President would choose, line-by-line, which parts of a bill to sign into law and which parts to veto.
However, the Constitution is explicit that if the President vetoes a law he must do so in toto; he may not veto a bill piecemeal.

Attempts, however, have been made to try to circumvent this restriction.
In the mid-1990's Congress tried to get around this restriction by giving the President the power to, effectively, veto \textit{portions} of a law after it went into effect.
The Supreme Court, however, rejected this as unconstitutional.  In \textit{Clinton v. City of New York}, 524 U.S. 417 (1998) the Supreme Court found that a post-enactment line-item veto amounted to a unilateral repeal of a validly-passed law by the Executive Branch.  As the Supreme Court pointed out, a validly-passed law can only be repealed by another validly-passed law -- and under the Constitution that would require the same Congressional Passage / Presidential Approval process as enacting the original law.

The Supreme Court held that it would violate the principle of separation of powers for the Executive branch to have the ability to make law (in this case via a power to repeal a validly-passed law), even if such power was supposedly bestowed on him by law by Congress.  The Constitution explicitly delegates the power to make law to Congress in Article I, and expressly limits the Executive branch's participation in the law-making process to the President's veto power.

As an interesting historical note, however, during the Civil War the Confederacy's Constitution gave its President a limited line-item veto with respect to appropriations.
The pertinent language was in Article I, Section 7 of the Confederate Constitution which, similar the United States Constitution, contained the provisions for making laws in the Confederacy.
The first part of the section read virtually identically to the United States Constitution, requiring passage of a law by both Houses of the Confederate Congress and presentment and signature by the President for a bill to become a law.
However, after detailing the veto, veto override, and pocket veto provisions, the Confederate Constitution added an additional clause:

\begin{quote}
The President may approve any appropriation and disapprove any other appropriation in the same bill. 
In such case he shall, in signing the bill, designate the appropriations disapproved; and shall return a copy of such appropriations, with his objections, to the House in which the bill shall have originated; and the same proceedings shall then be had as in case of other bills disapproved by the President.
\end{quote}

This provision in the Confederate Constitution gave the President of the Confederacy the power to veto \textit{individual} appropriations -- that is, the President had a line-item veto over spending.


